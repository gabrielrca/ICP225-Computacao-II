\title{Lista de exercicios: Revisão}
\author{Gabriel Rodrigues Caldas de Aquino}
\date{11/06/2025}

\begin{document}

\maketitle

\section{Conversão de string para número decimal}

\textbf{Descrição:}  
Peça ao usuário que digite um número com casas decimais (float). Utilize \textit{try/except} para capturar exceções no caso de o valor não poder ser convertido (como letras ou símbolos). Exiba o número com duas casas decimais após a conversão bem-sucedida.

\section{Remoção de elemento de lista}

\textbf{Descrição:}  
Crie uma lista de números e peça ao usuário que digite um valor para remover da lista. Utilize \textit{try/except} para tratar a exceção que ocorre caso o valor informado não exista na lista.

\section{Abertura e leitura de arquivos numéricos}

\textbf{Descrição:}  
Solicite ao usuário o nome de um arquivo que contenha números inteiros (um por linha). Tente abrir e ler os valores, convertendo cada linha em número. Utilize \textit{try/except} para tratar: (i) erro na abertura do arquivo; (ii) erro na conversão de alguma linha para inteiro.

\section{Divisão segura com múltiplas verificações}

\textbf{Descrição:}  
Solicite ao usuário dois valores. Tente converter ambos para números e realizar uma divisão. Trate com \textit{try/except} os seguintes erros: (i) erro de conversão; (ii) divisão por zero. Utilize \textit{else} para mostrar o resultado e \textit{finally} para imprimir uma mensagem de encerramento.

\section{Verificação de tipo e valor de entrada}

\textbf{Descrição:}  
Peça ao usuário que digite um número inteiro entre 1 e 100. Use \textit{try/except} para tratar entradas inválidas (não numéricas) e utilize \textit{raise} com \textit{ValueError} se o número estiver fora do intervalo permitido. Exiba uma mensagem adequada em cada caso.


\section{Validação de senha com exceções personalizadas}

\textbf{Descrição:}  
Crie uma função para validar uma senha fornecida pelo usuário. As regras são:
\begin{itemize}
    \item A senha deve ter pelo menos 8 caracteres.
    \item Deve conter ao menos um número.
    \item Deve conter ao menos uma letra maiúscula.
\end{itemize}

Se a senha não atender a uma das condições acima, lance uma exceção personalizada com uma mensagem indicando o motivo da falha na validação.


\section{Filtro de Palavras}

\textbf{Descrição:}  
Crie um programa que leia o conteúdo de um arquivo chamado \texttt{textao.txt} e grave em um novo arquivo \texttt{palavras-chave.txt} apenas as linhas que contenham uma determinada palavra-chave fornecida pelo usuário. 

\begin{itemize}
    \item Exemplo de palavra-chave: \texttt{segurança}
\end{itemize}

\section{Cópia de Arquivo Linha a Linha}

\textbf{Descrição:}  
Escreva um programa que copie o conteúdo de um arquivo chamado \texttt{origem.txt} para um novo arquivo chamado \texttt{copia.txt}, escrevendo uma linha por vez.

\begin{itemize}
    \item Certifique-se de fechar os arquivos ao final do processo.
\end{itemize}

\section{Contador de Palavras}

\textbf{Descrição:}  
Crie um programa que leia um arquivo \texttt{textao.txt} e conte o total de palavras no texto. O programa deve imprimir a contagem final na tela.

\begin{itemize}
    \item Considere que palavras estão separadas por espaço.
    \item Dica: use \texttt{split()} para separar as palavras de cada linha.
\end{itemize}

\section{Atualização de Preços em Arquivo CSV}

\textbf{Descrição:}  
Dado um arquivo \texttt{produtos.csv} no formato \texttt{Produto,Preco}, crie um programa que aplique um reajuste de 10\% em todos os preços e grave o resultado em \texttt{produtos-reajustados.csv}.

\begin{itemize}
    \item Exemplo de entrada:
\end{itemize}

\begin{verbatim}
Caneta,2.50
Lápis,1.00
Caderno,15.00
\end{verbatim}

\begin{itemize}
    \item Saída esperada:
\end{itemize}

\begin{verbatim}
Caneta,2.75
Lápis,1.10
Caderno,16.50
\end{verbatim}

\section{Leitura e cálculo de desvio padrão}

\textbf{Descrição:} Crie um arquivo \textit{dados.txt} contendo 6 números reais separados por espaço. Escreva um programa que:

\begin{enumerate}
    \item Leia o arquivo no modo \textit{r} e use \textit{split()} para obter os valores.
    \item Converta a lista de strings em um vetor NumPy de float.
    \item Calcule e exiba o desvio padrão dos valores.
\end{enumerate}

\textbf{Exemplo}:
\begin{verbatim}
import numpy as np
G = np.array([10.0, 12.0, 23.0, 23.0, 16.0, 23.0])
print(G.std())
#resultado: 4.898979485566356
\end{verbatim}

\section{Leitura de matriz e média por coluna}

\textbf{Descrição:} Crie um arquivo chamado \textit{notas.txt} contendo 4 linhas, com 3 números reais por linha, representando notas de alunos. Escreva um programa que:

\begin{enumerate}
    \item Leia todas as linhas e converta em uma matriz NumPy.
    \item Calcule a média de cada coluna (cada prova).
    \item Exiba o vetor com as médias.
\end{enumerate}

\textbf{Exemplo}:
\begin{verbatim}
H = np.array([[7.5, 8.0, 6.0],
              [6.0, 9.0, 7.0],
              [8.5, 7.5, 9.0],
              [9.0, 6.0, 8.0]])
print(H.mean(axis=0))
#resultado: [7.75 7.625 7.5]
\end{verbatim}

\section{Criação de matriz identidade e gravação}

\textbf{Descrição:} Escreva um programa que:

\begin{enumerate}
    \item Peça ao usuário um número inteiro \( n \).
    \item Crie uma matriz identidade \( n \times n \) usando NumPy.
    \item Grave essa matriz no arquivo \textit{identidade.txt}, com os números separados por espaço.
\end{enumerate}

\textbf{Exemplo}:
\begin{verbatim}
import numpy as np
I = np.identity(3, dtype=int)
print(I)
#resultado:
[[1 0 0]
 [0 1 0]
 [0 0 1]]
\end{verbatim}

\section{Diferença entre vetores de arquivos}

\textbf{Descrição:} Crie dois arquivos: \textit{vetor1.txt} e \textit{vetor2.txt}, cada um com 5 inteiros separados por espaço. Escreva um programa que:

\begin{enumerate}
    \item Leia os dois vetores a partir dos arquivos.
    \item Calcule a diferença absoluta entre os dois vetores (elemento a elemento).
    \item Grave o vetor resultante em \textit{diferenca.txt}.
\end{enumerate}

\textbf{Exemplo}:
\begin{verbatim}
import numpy as np
A = np.array([10, 20, 30, 40, 50])
B = np.array([8, 18, 25, 45, 60])
print(np.abs(A - B))
#resultado: [ 2  2  5  5 10]
\end{verbatim}

\section{Contagem de números pares e ímpares em vetor}

\textbf{Descrição:} Crie um arquivo chamado \textit{numeros.txt} contendo 15 números inteiros em uma única linha, separados por espaço. Faça um programa que:

\begin{enumerate}
    \item Leia o arquivo e crie um vetor NumPy com os valores.
    \item Conte quantos números pares e quantos ímpares existem no vetor.
    \item Exiba o resultado na tela.
\end{enumerate}

\textbf{Exemplo}:
\begin{verbatim}
import numpy as np
V = np.array([1, 2, 3, 4, 5, 6])
pares = np.sum(V % 2 == 0)
impares = np.sum(V % 2 != 0)
print("Pares:", pares, "Ímpares:", impares)
#resultado: Pares: 3 Ímpares: 3
\end{verbatim}


\end{document}

