\title{Lista de exercicios: Tratamento de Exceção}
\author{Gabriel Rodrigues Caldas de Aquino}
\date{\today}
\begin{document}


\maketitle

\section{Conversão de entrada para número inteiro}

\textbf{Descrição:}
Solicite ao usuário que digite um número inteiro. Utilize \textit{try/except} para capturar o erro caso o usuário insira um valor que não possa ser convertido para inteiro (por exemplo, letras ou símbolos).

\section{Calcular Divisão}

\textbf{Descrição}:
Peça ao usuário dois números e exiba o resultado da divisão do primeiro pelo segundo. Utilize \textit{try/except} para tratar a exceção de divisão por zero

\section{Acesso a elementos de lista}

\textbf{Descrição}:
Crie uma lista com alguns elementos e solicite ao usuário que digite um índice para acessar um elemento da lista. Utilize \textit{try/except} para tratar o erro caso o índice informado seja inválido (fora do intervalo da lista).

\section{Soma de elementos de uma lista}

\textbf{Descrição}:
Solicite ao usuário que insira uma lista de números separados por espaço. Converta os valores para inteiros e calcule a soma. Utilize \textit{try/except} para tratar possíveis erros durante a conversão.

\section{Leitura de arquivo}

\textbf{Descrição}:
Peça ao usuário o nome de um arquivo para leitura. Utilize \textit{try/except} para tratar a exceção caso o arquivo não exista.

\section{Cálculo de média com tratamento de exceções}

\textbf{Descrição}:
Solicite ao usuário que insira uma lista de números separados por vírgula. Converta os valores para números e calcule a média. Utilize \textit{try/except} para tratar: (i) possíveis erros durante a conversão; e (ii) divisão por zero.

\section{Leitura de arquivo com tratamento de múltiplas exceções}

\textbf{Descrição}:
Peça ao usuário o nome de um arquivo e tente abri-lo para leitura. Utilize \textit{try/except} para tratar a seguintes exceções: (i) caso o arquivo não existir; (ii) Adicione também um tratamento para qualquer outra exceção genérica.

\section{Validação de entrada com exceções específicas}

\textbf{Descrição}:
Solicite ao usuário que insira um número inteiro positivo. Utilize \textit{try/except} para capturar entradas inválidas e levante uma exceção personalizada (use o \textit{raise} com \textit{ValueError}) caso o número seja negativo ou zero.

\section{Manipulação de dicionários com tratamento de exceções}

\textbf{Descrição}:
Crie um dicionário com alguns pares chave-valor. Solicite ao usuário que insira uma chave e exiba o valor correspondente. Utilize \textit{try/except} para tratar o caso em que a chave não existe no dicionário.

\section{Uso de \textit{else} e \textit{finally} em blocos \textit{try/except}}

\textbf{Descrição}:
Crie um programa que solicite ao usuário dois números e exiba o resultado da divisão do primeiro pelo segundo. Utilize \textit{try/except} para capturar possíveis exceções, \textit{else} para exibir uma mensagem quando a operação for bem-sucedida e \textit{finally} para exibir uma mensagem final independente do resultado.


\pagebreak


\section{\textbf{BONUS}: Faça o tratamento dos possíveis erros que podem acontecer neste código mostrado em aula}

\begin{verbatim}
def calcular_media(valores):
    soma = sum(valores)
    media = soma / len(valores)
    return media

def ler_arquivo(nome_arquivo):
    with open(nome_arquivo, 'r') as file:
        dados = file.read().split(',')
    return [float(dado) for dado in dados]

print("=== Calculadora de Média ===")
valores = input("Digite números separados por vírgula: ").split(',')

valores = [int(x) for x in valores]

print(f"Valores de entrada: {valores}")

media = calcular_media(valores)
print(f"Média do teclado: {media}")

dados_arquivo = ler_arquivo('dados.txt')
print(f"Média do arquivo: {calcular_media(dados_arquivo)}")

\end{verbatim}

\end{document}
