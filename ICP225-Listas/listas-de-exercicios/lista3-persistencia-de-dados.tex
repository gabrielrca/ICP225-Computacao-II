\title{Lista de exercícios 3: Persistência de Dados}
\author{Gabriel Rodrigues Caldas de Aquino}
\date{28/05/2025}

\begin{document}

\maketitle
    
\section{Leitura Básica de Arquivo}
    \textbf{Descrição:} Crie um programa em Python que abra um arquivo chamado "dados.txt" (modo leitura), imprima todo o seu conteúdo na tela e depois feche o arquivo.

\section{Contador de Linhas}
        \textbf{Descrição:} Escreva um programa que conte quantas linhas existem em um arquivo chamado "nomes.txt". O programa deve abrir o arquivo, contar as linhas, imprimir o total e fechar o arquivo.

\section{Gravação Simples}
        \textbf{Descrição:} Crie um programa que grave três nomes (um em cada linha) em um arquivo chamado "alunos.txt". Use o modo de escrita ('w') e não esqueça de fechar o arquivo.

\section{Adição de Conteúdo}
        \textbf{Descrição:} Modifique o exercício anterior para adicionar mais dois nomes ao final do arquivo "alunos.txt" (em vez de sobrescrever). Use o modo de adição ('a').

\section{Tratamento de Erros}
        \textbf{Descrição:} Escreva um programa que tente abrir um arquivo "notas.txt" para leitura. Se o arquivo não existir, o programa deve imprimir uma mensagem amigável (como "Arquivo não encontrado") em vez de mostrar o erro do Python. Use try-except.   

\section{Manipulação de Arquivo CSV}

\textbf{Descrição:} Crie um programa que leia um arquivo \textit{dados.csv} (no formato Nome,Idade,Cidade) e gere um novo arquivo \texttt{maiores-de-idade.txt} contendo apenas as pessoas com idade maior ou igual a 18 anos.

\begin{itemize}
    \item Exemplo de entrada (\textit{dados.csv}):
\end{itemize}

\begin{verbatim}
João,17,Caxias
Maria,19,Maricá
Pedro,20,Niterói
Ana,16,Belém
\end{verbatim}
\begin{itemize}
    \item Saída esperada (\textit{maiores-de-idade.txt}):
\end{itemize}
\begin{verbatim}
Maria,19,Maricá
Pedro,20,Niterói
\end{verbatim}


\section{Busca em Arquivo}

\textbf{Descrição:} Crie um programa que permita ao usuário buscar um nome em um arquivo \textit{contatos.txt}, onde cada linha tem o formato \textit{Nome:Telefone}. Se o nome existir, exiba o telefone; caso contrário, informe que o contato não foi encontrado

\begin{itemize}
    \item Exemplo de entrada (\textit{contatos.txt}):
\end{itemize}

\begin{verbatim}
João:21987654321
Maria:21912345678
Pedro:21955554444
\end{verbatim}


\begin{itemize}
    \item Exemplo de execução:
\end{itemize}

\begin{verbatim}
Digite o nome a buscar: Maria
Telefone: 21912345678      
\end{verbatim}

\section{Log de Execução}

\textbf{Descrição:} Crie um programa que registre em um arquivo \textit{log.txt} todas as vezes que ele for executado, adicionando uma nova linha com a data e hora.

\begin{itemize}
    \item Formato:
\end{itemize}
    
\begin{verbatim}
Programa executado em: 2025-05-29 14:30:00  
Programa executado em: 2025-05-30 09:15:00      
\end{verbatim}


\textbf{Dica}: Use \textit{import datetime} e \textit{datetime.now()} para obter a data/hora atual

\section{Soma de Dados Numéricos}

Crie um programa que leia um arquivo \textit{valores.txt} (formato: \textit{NodeID,Valor}) e calcule:

\begin{itemize}
\item A soma total dos valores.

    \item O maior valor da lista.

    \item Grave os resultados em um novo arquivo estatisticas.txt.
\end{itemize}

\begin{itemize}
    \item Exemplo de entrada (\textit{valores.txt}):
\end{itemize}

\begin{verbatim}
Node1,15.5  
Node2,20.0  
Node3,10.2     
\end{verbatim}

\begin{itemize}
    \item Saída esperada (\textit{estatisticas.txt}):
\end{itemize}
\begin{verbatim}
Soma total: 45.7  
Maior valor: 20.0  
\end{verbatim}

\section{Operações com Matrizes a partir de Arquivos}

\textbf{Objetivo}:
Ler duas matrizes de arquivos \textit{.txt}, realizar operações básicas (soma, subtração) e salvar o resultado em um novo arquivo.

\begin{itemize}
    \item Utilize dois arquivos (\textit{matrizA.txt} e \textit{matrizB.txt}) no formato onde cada linha representa uma linha da matriz, com valores separados por vírgula:
\end{itemize}

\begin{verbatim}
1 2 3
4 5 6
7 8 9
\end{verbatim}

Operações Requeridas:
\begin{itemize}


       \item  Ler as matrizes dos arquivos

      \item   Verificar se pode ser realizada a operação

      \item   Realizar soma ou subtração

      \item   Salvar o resultado em resultado.txt
\end{itemize}
\end{document}