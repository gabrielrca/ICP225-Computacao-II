\title{Lista de Exercícios: Revisão Inicial}
\author{Gabriel Rodrigues Caldas de Aquino}
\date{22/05/2025}

\begin{document}

\maketitle

\section{Tipos de Dados e Operações}
Peça ao usuário dois números inteiros e exiba a soma, subtração, multiplicação e divisão entre eles.

\section{Manipulação de Strings}
Solicite ao usuário seu nome completo e exiba: o nome em letras maiúsculas, o número de caracteres (sem espaços) e apenas o primeiro nome.

\section{Uso de Funções}
Implemente uma função que receba a largura e altura de um retângulo e retorne sua área. No programa principal, peça os valores ao usuário e exiba o resultado.

\section{Listas}
Peça ao usuário cinco números, armazene-os em uma lista e exiba o maior e o menor valor.

\section{Dicionários}
Crie um dicionário com três produtos e seus preços. Permita que o usuário digite o nome de um produto e exiba o preço correspondente.

\section{Estruturas Condicionais}
Solicite ao usuário um número e informe se ele é positivo, negativo ou zero.

\section{Loops}
Utilizando um loop \texttt{for}, exiba todos os números pares de 0 a 20.

\section{Entrada e Saída de Dados}
Peça ao usuário seu nome e idade e exiba uma mensagem formatada dizendo “\texttt{<nome>} tem \texttt{<idade>} anos”.

\section{Trabalhando com Tuplas}
Crie uma tupla com os dias da semana. Solicite um número de 1 a 7 e exiba o dia correspondente.

\section{Operadores Lógicos}
Solicite a idade e a nacionalidade de uma pessoa e diga se ela pode votar (idade $\geq$ 16 e nacionalidade brasileira).


\section{\textbf{Bônus: Programa de Revisão Completa}}
Crie um programa que:
\begin{itemize}
\item Leia o nome, idade e três notas de um aluno.
\item Calcule a média das notas usando uma função.
\item Armazene os dados em um dicionário.
\item Exiba uma mensagem dizendo se o aluno foi aprovado (média $\geq$ 6) ou reprovado.
\end{itemize}



\section{Lista de compras com controle de estoque}

Crie um dicionário com produtos e seus estoques iniciais. Depois, peça que o usuário digite o nome de um produto e a quantidade desejada.
Se houver estoque suficiente, atualize o dicionário e exiba o estoque restante.
Caso contrário, informe que o produto está indisponível na quantidade solicitada.


\section{Estatísticas de notas}
Leia um número indeterminado de notas (o usuário digita “fim” para parar). Armazene as notas em uma lista e, ao final, exiba:

\begin{itemize}
    \item Quantidade de notas

    \item     Maior e menor nota

    \item     Média das notas (usando uma função)

    \item     Quantidade de notas acima da média

\end{itemize}

\section{Analisador de Texto}
Peça ao usuário que digite um texto qualquer.
O programa deve exibir:
\begin{itemize}
\item Quantas letras existem no texto
    
    \item Quantas palavras existem no texto.
\end{itemize}

    

    
\end{document}