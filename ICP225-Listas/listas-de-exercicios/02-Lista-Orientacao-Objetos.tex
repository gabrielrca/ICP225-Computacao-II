\title{Lista de Exercícios: Classes e Instâncias Básicas}
\author{Prof. Gabriel Rodrigues Caldas de Aquino}
\date{Compilado em: \\ \today}

\begin{document}

\maketitle

\section{Criação de Classe}
Crie uma classe \texttt{Monstro} com os atributos: \texttt{nome}, \texttt{vida} e \texttt{forca}. Adicione um método \texttt{apresentar()} que mostra esses valores.

\section{Método Construtor}
Implemente a classe \texttt{Arma} com \texttt{\_\_init\_\_} que recebe \texttt{nome} e \texttt{dano}. Crie duas instâncias: "Espada" (dano 15) e "Arco" (dano 10).

%\section{Métodos Simples}
%Na classe \texttt{Personagem} do slide, adicione um método \texttt{comemorar()} que aumenta a vida em 10 pontos e imprime "Comemoração!".

\section{Interação entre Objetos}
Crie um método \texttt{usar\_arma(self, arma, alvo)} na classe \texttt{Personagem} que reduz a vida do alvo pelo dano da arma.

\section{Método Especial \_\_str\_\_}
Modifique a classe \texttt{Personagem} para mostrar nome e vida quando imprimir o objeto (usando \texttt{\_\_str\_\_}).

\section{Lista de Objetos}
Crie 3 instâncias de \texttt{Personagem} com diferentes atributos e armazene numa lista. Percorra a lista chamando o método \texttt{apresentar()} de cada um.

\section{Atributo com Valor Padrão}
Adicione um atributo \texttt{nivel} na classe \texttt{Personagem} que começa com valor 1. Crie um método \texttt{subir\_nivel()} que incrementa esse valor.

\section{Validação Simples}
Modifique o método \texttt{defender()} para garantir que a vida nunca fique negativa (se ficar, defina como 0).

\section{Bônus: Sistema de Batalha}
Crie um loop onde:
\begin{itemize}
\item Seu personagem ataca um monstro
\item O monstro ataca de volta
\item Repete até alguém ter vida ≤ 0
\item Mostre o vencedor
\end{itemize}

\end{document}