\title{Lista de exercícios 2: Tratamento de Exceção}
\author{Gabriel Rodrigues Caldas de Aquino}
\date{27/05/2025}

\begin{document}

\maketitle

\section{Projeto de Controle de Vazão de Água}

Um sistema de controle de vazão de água precisa calcular o fluxo com base nos seguintes parâmetros fornecidos pelo usuário:

\begin{itemize}
    \item Área da Seção Transversal do Cano (m²).
    \item Velocidade do Fluxo (m/s).
\end{itemize}

A fórmula para o cálculo do fluxo é:

\begin{itemize}
    \item \textit{Q=A*V}
\end{itemize}


Onde: \begin{itemize}
    \item Q é o fluxo (em m³/s)
    \item A é a área (em m²)
    \item V é a velocidade (em m/s).
\end{itemize} 
O programa deve:
\begin{enumerate}
    \item Capturar os valores de entrada para A e V.
    \item Tratar as seguintes exceções:
    \begin{itemize}
        \item ValueError: Caso o usuário digite um valor não numérico.
        \item ZeroDivisionError: Caso a área da seção transversal A seja zero.
        \item Exceção Personalizada: Lançar uma exceção caso a área ou a velocidade sejam valores negativos.
    \end{itemize}
\end{enumerate}



\section{Exercicio}

Crie uma função que valida entradas numéricas e lança uma exceção personalizada se o número não estiver no intervalo [1, 100].


\section{Exercicio}

Crie uma função para validar uma senha fornecida pelo usuário. As regras são:
\begin{itemize}
    \item A senha deve ter pelo menos 8 caracteres.
    \item Deve conter ao menos um número.
    \item Deve conter ao menos uma letra maiúscula.
\end{itemize}

Se a senha não atender a uma das condições acima:
\begin{itemize}
    \item Lançar uma exceção personalizada com uma mensagem indicando o erro.
\end{itemize}


\end{document}
