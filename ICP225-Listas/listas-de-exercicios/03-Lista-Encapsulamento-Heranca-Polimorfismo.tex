\documentclass{article}
\usepackage[utf8]{inputenc}
\usepackage{graphicx}

\title{Lista de Exercícios: Herança, Polimorfismo e Encapsulamento}
\author{Prof. Gabriel Rodrigues Caldas de Aquino}
\date{Compilado em: \\ \today}

\begin{document}

\maketitle

\section{Itens de empréstimo}
Crie a classe \texttt{Item} com os atributos privados: \texttt{título}, \texttt{nomePessoa}, \texttt{telefonePessoa} e 
\texttt{dataEmprestimo}, com seus getters (função para consultar valor) e setters (função para colocar valor) públicos.

\begin{enumerate}
    \item Crie a classe \texttt{Livro} e a classe \texttt{CD}, que herdem de \texttt{Item} e tenham os atributos 
    \texttt{autor} e \texttt{artista}, respectivamente, também privados, com getters e setters públicos. 
    \item Livros devem ter número de páginas
    \item CDs devem ter quantidade de faixas
    \item Em uma lista, coloque pelo menos 3 itens diferentes de cada tipo, emprestados para pessoas diferentes. 
    \item Percorra essa lista imprimindo o que foi emprestado, para quem e quando e os seus atributos 



\end{enumerate}


\section{Formas Geométricas}
Crie a classe \texttt{Forma} com os atributos privados: \texttt{nome}, \texttt{cor}, \texttt{espessuraBorda} e
\texttt{dataCriacao}, com seus getters (função para consultar valor) e setters (função para colocar valor) públicos.

\begin{enumerate}
\item Crie a classe \texttt{Retangulo} e a classe \texttt{Circulo}, que herdem de \texttt{Forma} e tenham os atributos
\texttt{largura} e \texttt{altura} para Retângulo, e \texttt{raio} para Círculo, também privados, com getters emétodo abstrato chamado exibirDetalhes(), que será implementado pelas classes filhas para mostrar informações específicas do item setters públicos.
\item Retângulos devem ter um método para calcular área (largura × altura)
\item Círculos devem ter um método para calcular área ($\pi \times \textrm{raio}^2$)
\item Retângulos devem ter um método para calcular perímetro (2 × (largura + altura))
\item Círculos devem ter um método para calcular perímetro ($2 \times \pi \times raio$)
\item Crie diferentes formas geométricas
\item Sinta-se livre para criar formas que você quiser

\end{enumerate}

\section{Funcionários e Departamentos}
Crie a classe \texttt{Funcionario} com os atributos privados: \texttt{nome}, \texttt{cpf}, \texttt{telefone} e 
\texttt{salario}, com seus getters (função para consultar valor) e setters (função para colocar valor) públicos.

\begin{enumerate}
    \item Crie a classe \texttt{Gerente} e a classe \texttt{Estagiario}, que herdem de \texttt{Funcionario}. 
    \item Gerentes devem ter o atributo privado \texttt{departamento} e estagiários devem ter o atributo privado \texttt{universidade}, ambos com getters e setters públicos.
    \item Gerentes devem ter um método para escrever seu  \texttt{nome}, \texttt{cpf}, \texttt{telefone}, 
\texttt{salario} e \texttt{departamento}.
    \item Estagiários devem ter um método para escrever seu  \texttt{nome}, \texttt{cpf}, \texttt{telefone}, 
\texttt{salario} e \texttt{universidade}.
\item Utilize o polimorfismo
   
\end{enumerate}


\section{Veículos}
Crie a classe \texttt{Veiculo} com os atributos privados: \texttt{marca}, \texttt{modelo}, \texttt{ano} e 
\texttt{preco}, com seus getters (função para consultar valor) e setters (função para colocar valor) públicos.

\begin{enumerate}
    \item Crie a classe \texttt{Carro} e a classe \texttt{Moto}, que herdem de \texttt{Veiculo}. 
    \item Carros devem ter o atributo privado \texttt{quantidadePortas}, enquanto motos devem ter o atributo privado \texttt{cilindrada}, ambos com getters e setters públicos.
    \item Carros devem ter um método para calcular o valor do seguro (5\% do preço).
    \item Motos devem ter um método para calcular o valor do seguro (3\% do preço).
    
\end{enumerate}

\section{Animais}
Crie a classe \texttt{Animal} com os atributos privados: \texttt{nome} e \texttt{idade}, com seus getters e setters públicos, e um método \texttt{emitir\_som()} que imprime um som genérico.

\begin{enumerate}
    \item Crie as classes \texttt{Cachorro}, \texttt{Gato} e \texttt{Pato}, que herdem de \texttt{Animal} e sobrescrevam o método \texttt{emitir\_som()} para imprimir sons específicos: "auau", "miau" e "quack", respectivamente.
    \item Crie uma lista contendo pelo menos uma instância de cada tipo de animal.
    \item Percorra a lista chamando o método \texttt{emitir\_som()} de cada animal, demonstrando polimorfismo.
    \item Crie um método adicional \texttt{apresentar()} na classe \texttt{Animal} que imprime o nome e a idade, e sobrescreva esse método nas subclasses para incluir uma mensagem específica de cada animal.
\end{enumerate}
















\end{document}
