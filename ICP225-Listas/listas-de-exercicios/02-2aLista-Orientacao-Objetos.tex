\title{Lista de Exercícios: Definições básicas de classes}
\author{Prof. Gabriel Rodrigues Caldas de Aquino}

\date{Compilado em: \\ \today}

\begin{document}

\maketitle

\section{Classe Animal}
Crie uma classe \texttt{Animal} com:
\begin{itemize}
\item Atributos: \texttt{nome}, \texttt{especie}, \texttt{idade}
\item Método \texttt{emitir\_som()} que imprime um som genérico
\item Método \texttt{envelhecer()} que aumenta a idade em 1 ano
\end{itemize}

\section{Classe Retângulo}
Implemente uma classe \texttt{Retangulo} com:
\begin{itemize}
\item Atributos: \texttt{base}, \texttt{altura}
\item Métodos: \texttt{calcular\_area()} e \texttt{calcular\_perimetro()}
\item Método \texttt{\_\_str\_\_} que mostra as dimensões
\end{itemize}

\section{Classe Conta Bancária}
Desenvolva uma classe \texttt{ContaBancaria} com:
\begin{itemize}
\item Atributos: \texttt{titular}, \texttt{saldo} (inicia com 0)
\item Métodos: \texttt{depositar(valor)}, \texttt{sacar(valor)} e \texttt{consultar\_saldo()}
\item Valide para não permitir saque maior que o saldo
\end{itemize}

\section{Classe Produto}
Crie uma classe \texttt{Produto} para sistema de estoque:
\begin{itemize}
\item Atributos: \texttt{codigo}, \texttt{nome}, \texttt{preco}, \texttt{quantidade\_estoque}
\item Métodos: \texttt{aplicar\_desconto(percentual)} e \texttt{vender(quantidade)}
\end{itemize}

\section{Classe Carro}
Implemente uma classe \texttt{Carro} com:
\begin{itemize}
\item Atributos: \texttt{marca}, \texttt{modelo}, \texttt{ano}, \texttt{quilometragem}
\item Métodos: \texttt{dirigir(km)} (aumenta quilometragem) e \texttt{avaliar\_estado()} (retorna "novo", "semi-novo" ou "usado" baseado na km)
\end{itemize}

\section{Classe Aluno}
Crie uma classe \texttt{Aluno} para sistema escolar:
\begin{itemize}
\item Atributos: \texttt{matricula}, \texttt{nome}, \texttt{notas} (lista)
\item Métodos: \texttt{adicionar\_nota(nota)}, \texttt{calcular\_media()} e \texttt{situacao()} (aprovado se média ≥ 6)
\end{itemize}

\section{Classe Contato}
Desenvolva uma classe \texttt{Contato} para agenda telefônica:
\begin{itemize}
\item Atributos: \texttt{nome}, \texttt{telefone}, \texttt{email}
\item Método \texttt{exibir\_contato()} que mostra todos os dados formatados
\item Método \texttt{atualizar\_telefone(novo\_numero)}
\end{itemize}

\section{Bônus: Classe Jogo}
Implemente uma classe \texttt{PersonagemJogo} com:
\begin{itemize}
\item Atributos: \texttt{nome}, \texttt{vida}, \texttt{energia}, \texttt{habilidades} (lista de strings)
\item Métodos: \texttt{aprender\_habilidade(nova\_habilidade)}, \texttt{usar\_habilidade()} (consome energia) e \texttt{descansar()} (recupera energia)
\end{itemize}

\pagebreak

{\huge Parte 2: Utilize como base os exemplos apresentados em aula para os exercícios seguintes}


\section{Criação de Classe}
Crie uma classe \texttt{Monstro} com os atributos: \texttt{nome}, \texttt{vida} e \texttt{forca}. Adicione um método \texttt{apresentar()} que mostra esses valores.

\section{Método Construtor}
Implemente a classe \texttt{Arma} com \texttt{\_\_init\_\_} que recebe \texttt{nome} e \texttt{dano}. Crie duas instâncias: "Espada" (dano 15) e "Arco" (dano 10).

%\section{Métodos Simples}
%Na classe \texttt{Personagem} do slide, adicione um método \texttt{comemorar()} que aumenta a vida em 10 pontos e imprime "Comemoração!".

\section{Interação entre Objetos}
Crie um método \texttt{usar\_arma(self, arma, alvo)} na classe \texttt{Personagem} que reduz a vida do alvo pelo dano da arma.

\section{Método Especial \_\_str\_\_}
Modifique a classe \texttt{Personagem} para mostrar nome e vida quando imprimir o objeto (usando \texttt{\_\_str\_\_}).

\section{Lista de Objetos}
Crie 3 instâncias de \texttt{Personagem} com diferentes atributos e armazene numa lista. Percorra a lista chamando o método \texttt{apresentar()} de cada um.

\section{Atributo com Valor Padrão}
Adicione um atributo \texttt{nivel} na classe \texttt{Personagem} que começa com valor 1. Crie um método \texttt{subir\_nivel()} que incrementa esse valor.

\section{Validação Simples}
Modifique o método \texttt{defender()} para garantir que a vida nunca fique negativa (se ficar, defina como 0).

\section{Bônus: Sistema de Batalha}
Crie um loop onde:
\begin{itemize}
\item Seu personagem ataca um monstro
\item O monstro ataca de volta
\item Repete até alguém ter vida ≤ 0
\item Mostre o vencedor
\end{itemize}

\end{document}