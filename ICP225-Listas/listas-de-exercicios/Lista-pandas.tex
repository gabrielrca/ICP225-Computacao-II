\title{Lista de exercicios: Pandas}
\author{Gabriel Rodrigues Caldas de Aquino}
\date{\today}
\begin{document}


\maketitle


\section{Análise de Medalhas Olímpicas}
\textbf{Use a base de dados}: \url{https://raw.githubusercontent.com/MainakRepositor/Datasets/refs/heads/master/Tokyo-Olympics/Medals.csv}



\textbf{Faça a seguinte análise de dados:}

\begin{enumerate}
    \item Mostre os países que ganharam mais de 5 medalhas de ouro.
    \item Ordene esses países pelo número de medalhas de ouro em ordem decrescente.
    \item Mostre os países que ganharam menos de 5 medalhas no total.
    \item Ordene esses países pelo número de medalhas de ouro em ordem crescente.
    \item Filtre apenas os registros do Brasil.
    \item Determine qual tipo de medalha (ouro, prata ou bronze) o Brasil ganhou \textbf{menos}.
    \item Crie um gráfico barra para mostrar a distribuição de medalhas (ouro, prata, bronze) do Brasil por modalidade.
\end{enumerate}


\section{Análise de apartamentos no Rio de Janeiro}
\textbf{Use a base de dados}: \url{https://raw.githubusercontent.com/mvinoba/notebooks-for-binder/refs/heads/master/dados.csv}



\begin{enumerate}
    \item Crie uma lista dos bairros que possuem entradas na base de dados (liste somente uma única ocorrência).
    \item Conte a quantidade de ocorrências de imóveis em cada bairro na base de dados.
    \item Filtre os apartamentos de Botafogo com condomínio mais barato do que R\$500,00
    \item Liste os apartamentos de Ipanema que tem mais de 2 quartos.
    \item Liste os apartamentos da Tijuca que tem mais de 1 quarto e condomínio menor que R\$700,00.
    \item Apresente um gráfico Numpy com dados relevantes sobre apartamentos de Copacabana (Apresente a sua análise).
\end{enumerate}


\section{Alunos do CEFET-MG}

Considere a base de dados em \url{https://dados.gov.br/dados/conjuntos-dados/12-alunos}

Pegue o conjunto de dados através da URL \url{https://www.dados.cefetmg.br/wp-content/uploads/sites/248/2023/02/PDA_2022-2024_1.2_Alunos_Anonimo.csv}

\textbf{Obs:} Note que este arquivo CSV utiliza ponto e vírgula (;) como separador e possui linhas de metadados no cabeçalho que precisam ser ignoradas - parâmetro skiprows
\begin{verbatim}
    df_cefet = pd.read_csv(url_cefet, sep=';', skiprows=2, encoding='latin1')
\end{verbatim}


\begin{enumerate}
    \item Carregue os dados corretamente, ignorando as linhas iniciais que não fazem parte do cabeçalho (linhas 1 e 2 do arquivo original).
    \item Conte quantos registros existem para "Ingressante" e quantos para "Concluinte".
    \item Liste os nomes de todos os \textbf{Campus} únicos presentes na base de dados.
    \item Filtre apenas os alunos do "CAMPUS NOVA SUÍÇA".
    \item Dentre os alunos do Campus Nova Suíça filtrados anteriormente, mostre apenas os que estão no curso de "TÉCNICO EM MECATRÔNICA".
    \item Identifique qual é o "Nível de Ensino" (Técnico, Graduação, etc.) que possui a maior quantidade de alunos registrados nesta base.
    \item Crie um gráfico de barras horizontais mostrando a quantidade de alunos por Campus (ordene do campus com mais alunos para o com menos alunos).
\end{enumerate}

\end{document}
