\title{Lista de Exercícios: Introdução ao Matplotlib}
\author{Gabriel Rodrigues Caldas de Aquino}
\date{03/07/2025}

\begin{document}

\maketitle

\section{Preço por kg de leite em pó}

\textbf{Descrição:}  

Você possui o arquivo de texto chamado \texttt{leite-preco-kg.txt}, contendo informações de diferentes marcas de leite em pó. Cada linha possui o nome do produto, o preço total do pacote em reais e o peso do pacote em kg, separados por vírgula.  

\begin{verbatim}
Leite em pó A,38.90,2.0
Leite em pó B,22.50,1.0
Leite em pó C,18.00,0.8
Leite em pó D,45.20,2.5
Leite em pó E,12.90,0.4
Leite em pó Z,12.90,0.0
Leite em pó F,27.50,1.2
Leite em pó G,55.00,3.0
\end{verbatim}

\textbf{Tarefas:}

\begin{itemize}
  \item Ler o arquivo linha por linha, separando corretamente os campos.
  \item Calcular o preço por kg de cada produto (preço dividido pelo peso).
  \item Gerar um gráfico de barras com \texttt{matplotlib} mostrando:
    \begin{itemize}
      \item eixo X: nome do produto
      \item eixo Y: preço por kg calculado
    \end{itemize}
  \item Tratar possíveis erros (por exemplo, divisão por zero) com \texttt{try-except}.
\end{itemize}

O gráfico deve conter título, legendas nos eixos e as barras identificadas.


\section{Cálculo de vendas com produto matricial}

\textbf{Descrição:}  

Você é responsável pelo controle de vendas de três produtos (Lápis, Caderno e Borracha) em três lojas diferentes (Loja A, Loja B e Loja C). Foi montada uma planilha com a quantidade de produtos vendidos em cada loja, conforme abaixo (cada linha representa uma loja, e cada coluna representa um produto):

\begin{verbatim}
        Lápis   Caderno   Borracha
Loja A   100       50        200
Loja B    80       60        150
Loja C   120       40        100
\end{verbatim}

Além disso, você possui os preços unitários de cada produto:

\begin{itemize}
    \item Lápis: R\$ 2,00
    \item Caderno: R\$ 10,00
    \item Borracha: R\$ 1,50
\end{itemize}

\textbf{Tarefa:}

Utilizando o \texttt{NumPy} e a operação de produto matricial, calcule o valor total recebido em cada loja considerando todas as vendas.


\section{Cálculo de vendas com preços diferenciados por loja}

\textbf{Descrição:}  

Você é responsável por três lojas (Loja A, Loja B e Loja C) que vendem três produtos (Lápis, Caderno e Borracha). Cada loja tem seu próprio preço para cada produto, e você já possui o levantamento das vendas conforme abaixo:

\textbf{Quantidade de produtos vendidos:}

\begin{verbatim}
        Lápis   Caderno   Borracha
Loja A   100       50        200
Loja B    80       60        150
Loja C   120       40        100
\end{verbatim}

\textbf{Preços dos produtos em cada loja:}

\begin{verbatim}
        Lápis   Caderno   Borracha
Loja A   2.00      9.50       1.60
Loja B   2.20     10.00       1.40
Loja C   1.90      9.80       1.50
\end{verbatim}

\textbf{Tarefa:}

Utilizando o \texttt{NumPy}, calcule o valor total recebido em cada loja, considerando a multiplicação elemento a elemento entre quantidade e preço, e depois somando os valores de cada loja.


\section{Soma vetorial de forças}

\textbf{Descrição:}  

Um objeto está sendo puxado por duas forças representadas diretamente em coordenadas cartesianas:

\begin{itemize}
    \item \textbf{F\textsubscript{1} = (10, 5) N}
    \item \textbf{F\textsubscript{2} = (–4, 12) N}
\end{itemize}

\textbf{Tarefa:}

\begin{quote}
Calcule a força resultante (soma vetorial) usando a biblioteca \texttt{NumPy}.
\end{quote}


\section{Plotando funções definidas pelo usuário}

\textbf{Descrição:}  

Em Python, crie uma função chamada \texttt{f(x)} onde você mesmo(a) define a função matemática (por exemplo, \(f(x) = x^2 + 3x + 1\)) dentro dela.

\textbf{Tarefas:}

\begin{itemize}
    \item Implemente a função \texttt{f(x)} contendo a expressão matemática desejada dentro dela.
    \item Use o \texttt{matplotlib} para plotar o gráfico de \texttt{f(x)} para valores de \(x\) no intervalo \([-10, 10]\), com pelo menos 100 pontos.
    \item Modifique a expressão dentro de \texttt{f(x)} para testar diferentes funções e verifique os gráficos resultantes.
    \item Utilize o \texttt{NumPy} para gerar os valores de \(x\).
\end{itemize}


\section{ Faça o tratamento dos possíveis erros que podem acontecer neste código mostrado em aula}

\begin{verbatim}
def calcular_media(valores):
    soma = sum(valores)
    media = soma / len(valores)
    return media

def ler_arquivo(nome_arquivo):
    with open(nome_arquivo, 'r') as file:
        dados = file.read().split(',')
    return [float(dado) for dado in dados]

print("=== Calculadora de Média ===")
valores = input("Digite números separados por vírgula: ").split(',')

valores = [int(x) for x in valores]

print(f"Valores de entrada: {valores}")

media = calcular_media(valores)
print(f"Média do teclado: {media}")

dados_arquivo = ler_arquivo('dados.txt')
print(f"Média do arquivo: {calcular_media(dados_arquivo)}")

\end{verbatim}

\section{Manipulação de arquivos em Python}

\textbf{Descrição:}  

Analise o seguinte código em Python:

\begin{verbatim}
with open("dados.txt", "w+") as arquivo:
    arquivo.write("Primeira linha\n")
    arquivo.seek(0)
    print(arquivo.read())
\end{verbatim}

\textbf{Perguntas:}

\begin{enumerate}
    \item[a)] O que será exibido no console após a execução do código?
    \item[b)] Qual seria o resultado se a linha \texttt{arquivo.seek(0)} fosse removida? Justifique sua resposta.
\end{enumerate}


\section{Uso do método \texttt{seek()} em arquivos}

\textbf{Descrição:}  

O método \texttt{seek()} reposiciona o cursor de leitura/escrita de um arquivo para um deslocamento específico, permitindo que operações subsequentes sejam feitas a partir de uma posição arbitrária dentro do arquivo.

\textbf{Exemplo:}

Considere que um arquivo chamado \texttt{letras.txt} contém o seguinte conteúdo:

\begin{verbatim}
ABCDEF
\end{verbatim}

E que o seguinte código seja executado em Python:

\begin{verbatim}
arq = open("letras.txt", "a+")
arq.seek(3)
print(arq.read())
arq.close()
\end{verbatim}

\textbf{Pergunta:}

\begin{itemize}
    \item Qual será a saída exibida no console após a execução do código? Explique detalhadamente o motivo.
\end{itemize}

\section{Verificação de triângulo equilátero}

\textbf{Descrição:}  

Considere o seguinte código em Python:

\begin{verbatim}
def verifica_equilatero(triangulo):
    if triangulo[0] == triangulo[1] == triangulo[2]:
        return True
    else:
        raise ValueError("Não é um triângulo equilátero!")
\end{verbatim}

\textbf{Perguntas:}

\begin{enumerate}
    \item[a)] O que acontece se for passado os valores \texttt{[4,5,4]} como argumento?
    \item[b)] Escreva um código que chame essa função e trate a exceção de forma adequada, exibindo a mensagem de erro para o usuário.
\end{enumerate}


\section{Tratamento de exceções em Python}

\textbf{Descrição:}  

Considere o seguinte código Python:

\begin{verbatim}
try:
    arquivo = open("dados.txt", "r")
    dados = arquivo.read()
except FileNotFoundError:
    print("Arquivo não encontrado!")
finally:
    print("Fim da execução")
\end{verbatim}

Suponha que o arquivo \texttt{dados.txt} não exista no diretório onde o código é executado.

\textbf{Perguntas:}

\begin{enumerate}
    \item[a)] Qual será a saída exibida na tela?
    \item[b)] Explique.
\end{enumerate}



\end{document}
