\title{Lista de exercícios 4: Persistência de Dados e Numpy}
\author{Prof. Gabriel Rodrigues Caldas de Aquino}
\date{05/06/2025}

\begin{document}

\maketitle

\section{Leitura de arquivo para vetor NumPy}

\textbf{Descrição:} Crie um arquivo chamado \textit{valores.txt} contendo 10 números inteiros separados por espaço (todos na mesma linha). Depois, escreva um programa que:
\begin{enumerate}
    \item Leia o conteúdo do arquivo usando o \textit{modo r}.
    \item Use split() para separar os números.
    \item Converta essa lista em um vetor NumPy de inteiros
    \item Exiba o vetor e sua soma
\end{enumerate}

\textbf{Exemplo}:
\begin{verbatim}
import numpy as np
a = np.array([1,2,3,4,5])
print(a)
#resultado: [1 2 3 4 5]
print(a.sum())
#resultado: 15
\end{verbatim}

\section{Escrita com \textit{w} e leitura com \textit{r}}

\textbf{Descrição:} Escreva um programa que:

\begin{enumerate}
    \item Gere 20 números inteiros aleatórios entre 1 e 100 e salve todos no arquivo \textit{random.txt} em uma única linha, separados por espaço, usando o modo \textit{w}.
    \item Depois, abra o mesmo arquivo no modo \textit{r}, leia a linha, use \textit{split()} e crie um vetor NumPy com esses números.
    \item Exiba o maior e o menor valor do vetor.
\end{enumerate}

\textbf{Exemplo}:
\begin{verbatim}
import numpy as np
B = np.random.randint(0, 100, size=20)
print(B)
#resultado: [11 98  6 57 74 10 47 64 46 34  1 74 37 28 80 89 14 99 63 29]
print(B.min())
#resultado: 1
print(B.max())
#resultado: 99
\end{verbatim}


\section{Leitura e escrita com \textit{r+}}

\textbf{Descrição:} Crie um arquivo \textit{dados.txt} com 5 números reais separados por espaço. Faça um programa que:

\begin{enumerate}
    \item  Abra o arquivo no modo \textit{r+} e leia os dados.
    \item  Converta os dados em um vetor NumPy de float.
    \item  Calcule a média dos números.
    \item Ao final do arquivo, escreva a seguinte linha: \textit{Media: X}, onde X é a média calculada.
\end{enumerate}

\textbf{Exemplo}:
\begin{verbatim}
import numpy as np  
C = np.array([3.3,4.2,1,7])
print(C)
#resultado: [3.3 4.2 1.  7. ]
print(C.mean())
#resultado: 3.875
\end{verbatim}

\section{Manipulação de arquivos e vetores}

\textbf{Descrição:} Crie um arquivo entrada.txt com 8 números inteiros, um por linha. Escreva um programa que:

\begin{enumerate}
    \item Leia cada linha do arquivo, remova o \textbackslash n, e crie um vetor NumPy com esses números.
    \item Calcule o produto de todos os números.
    \item Grave o resultado no arquivo saida.txt.
\end{enumerate}

\textbf{Exemplo}:
\begin{verbatim}
import numpy as np  
D =  np.array([1, 2, 3, 4, 5])
print(D.prod())
#resultado: 120
\end{verbatim}

\section{Escrita e multiplicação com NumPy}

\textbf{Descrição:} Faça o seguinte programa:

\begin{enumerate}
    \item Gere 10 números inteiros aleatórios entre 1 e 10.
    \item Salve os números no arquivo \textit{numeros.txt}, separados por espaço, usando o modo \textit{w}.
    \item Leia o arquivo com o modo \textit{r}, use \textit{split()} e crie um vetor NumPy.
    \item Multiplique cada elemento do vetor por 3 e exiba o novo vetor.
\end{enumerate}

\textbf{Exemplo}:
\begin{verbatim}
E = np.array([1, 2, 3, 4, 5])
print(E * 3)
#resultado: [ 3  6  9 12 15]
\end{verbatim}

\section{Soma de linhas e colunas}

\textbf{Descrição:} Crie um arquivo \textit{matriz.txt} com 3 linhas e 4 números inteiros por linha, separados por espaço.

\begin{enumerate}
    \item Abra o arquivo com o modo \textit{r}, leia todas as linhas.
    \item Para cada linha, use \textit{split()} e converta em um matriz NumPy.
    \item Some todos os vetores para obter a soma de cada coluna e exiba o vetor soma.
    \item Exiba o maior e menor valor por coluna
    \item Exiba o maior e menor valor por linha
\end{enumerate}

\textbf{Exemplo}:
\begin{verbatim}
F = np.array([[1, 2, 3, 4], [5, 6, 7, 8], [9, 10, 11, 12]])
print(F)
#resultado: 
[[ 1  2  3  4]
 [ 5  6  7  8]
 [ 9 10 11 12]]
soma_linhas = np.sum(F, axis=1)
print(soma_linhas)
#resultado: [10 26 42]
soma_colunas = np.sum(F, axis=0)
print(soma_colunas)
#resultado: [15 18 21 24]
maior_linhas = np.max(F, axis=1)
print(maior_linhas)
#resultado: [ 4  8 12]
menor_colunas = np.min(F, axis=0)
print(menor_colunas)
#resultado: [1 2 3 4]
\end{verbatim}

\end{document}