\documentclass{article}
\usepackage[utf8]{inputenc}
\usepackage{graphicx}
 \usepackage{amsmath}

\title{Lista de Exercícios: Classe abstrata}
\author{Prof. Gabriel Rodrigues Caldas de Aquino}
\date{Compilado em: \\ \today}

\begin{document}

\maketitle

\section{Itens de Empréstimo}

Crie um sistema para gerenciar itens que podem ser emprestados em uma biblioteca.

\begin{enumerate}
    \item Crie uma \textbf{classe abstrata} chamada \texttt{Item}.  
    Ela deve ter os seguintes atributos:
    \begin{itemize}
        \item \texttt{titulo} -- título do item,
        \item \texttt{nomePessoa} -- nome da pessoa que pegou emprestado,
        \item \texttt{dataEmprestimo} -- data em que o item foi emprestado.
    \end{itemize}
    Além disso, a classe deve possuir um \textbf{método abstrato} chamado \texttt{exibirDetalhes()}, 
    que será implementado pelas classes filhas para mostrar informações específicas de cada tipo de item.

    \item Crie duas classes concretas que herdem de \texttt{Item}:
    \begin{enumerate}
        \item \textbf{\texttt{Livro}}: deve conter os atributos:
        \begin{itemize}
            \item \texttt{autor},
            \item \texttt{numPaginas} -- número de páginas do livro.
        \end{itemize}

        \item \textbf{\texttt{CD}}: deve conter os atributos:
        \begin{itemize}
            \item \texttt{artista},
            \item \texttt{numFaixas} -- quantidade de faixas do CD.
        \end{itemize}
    \end{enumerate}

    \item Crie pelo menos \textbf{3 livros} e \textbf{3 CDs}, cada um emprestado para pessoas diferentes, 
    e coloque todos os itens em uma única lista.

    \item Percorra essa lista e imprima, para cada item:
    \begin{itemize}
        \item o tipo do item,
        \item o título,
        \item o nome da pessoa que pegou emprestado,
        \item a data do empréstimo,
        \item e os atributos específicos do item (como autor e número de páginas, ou artista e quantidade de faixas).
    \end{itemize}
\end{enumerate}

\section{Formas Geométricas}

Crie um sistema para representar formas geométricas utilizando herança e polimorfismo.

\begin{enumerate}
    \item Crie uma \textbf{classe abstrata} chamada \texttt{Forma}.  
    Ela deve ter os seguintes atributos:
    \begin{itemize}
        \item \texttt{nome} -- nome da forma geométrica,
        \item \texttt{cor} -- cor da forma,
        \item \texttt{espessuraBorda} -- espessura da borda,
        \item \texttt{dataCriacao} -- data em que a forma foi criada.
    \end{itemize}
    
    Além disso, a classe deve possuir um \textbf{método abstrato} chamado \texttt{exibirDetalhes()}, 
    que será implementado pelas classes filhas para mostrar informações específicas de cada forma.

    \item Crie duas classes concretas que herdem de \texttt{Forma}:
    \begin{enumerate}
        \item \textbf{\texttt{Retangulo}}: deve conter os atributos:
        \begin{itemize}
            \item \texttt{largura},
            \item \texttt{altura}.
        \end{itemize}
        
        Também deve possuir:
        \begin{itemize}
            \item Um método para calcular a \textbf{área}: \(\text{largura} \times \text{altura}\),
            \item Um método para calcular o \textbf{perímetro}: \(2 \times (\text{largura} + \text{altura})\).
        \end{itemize}

        \item \textbf{\texttt{Circulo}}: deve conter o atributo:
        \begin{itemize}
            \item \texttt{raio}.
        \end{itemize}
        
        Também deve possuir:
        \begin{itemize}
            \item Um método para calcular a \textbf{área}: \(\pi \times \text{raio}^2\),
            \item Um método para calcular o \textbf{perímetro}: \(2 \times \pi \times \text{raio}\).
        \end{itemize}
    \end{enumerate}

    \item Crie diferentes formas geométricas (pelo menos 2 retângulos e 2 círculos) 
    e armazene todas em uma única lista.

    \item Percorra a lista e utilize o método \texttt{exibirDetalhes()} para mostrar, para cada forma:
    \begin{itemize}
        \item nome,
        \item cor,
        \item espessura da borda,
        \item data de criação,
        \item tipo da forma,
        \item medidas específicas (largura/altura ou raio),
        \item área e perímetro calculados.
    \end{itemize}

    \item Sinta-se livre para adicionar outros tipos de formas, se desejar.
\end{enumerate}

\section{Animais}

Crie um sistema para representar diferentes tipos de animais utilizando herança e polimorfismo.

\begin{enumerate}
    \item Crie uma \textbf{classe abstrata} chamada \texttt{Animal}.  
    Ela deve ter os seguintes atributos:
    \begin{itemize}
        \item \texttt{nome} -- nome do animal,
        \item \texttt{idade} -- idade do animal.
    \end{itemize}

    A classe deve possuir:
    \begin{itemize}
        \item Um \textbf{método abstrato} chamado \texttt{emitir\_som()}, que será implementado pelas subclasses para exibir o som específico de cada animal.
        \item Um método \texttt{apresentar()} que imprime o nome e a idade do animal.
    \end{itemize}

    \item Crie três subclasses que herdem de \texttt{Animal}:
    \begin{enumerate}
        \item \textbf{\texttt{Cachorro}} -- o método \texttt{emitir\_som()} deve imprimir \texttt{"auau"}.
        \item \textbf{\texttt{Gato}} -- o método \texttt{emitir\_som()} deve imprimir \texttt{"miau"}.
        \item \textbf{\texttt{Pato}} -- o método \texttt{emitir\_som()} deve imprimir \texttt{"quack"}.
    \end{enumerate}

    Além disso, cada subclasse deve sobrescrever o método \texttt{apresentar()} para incluir uma
    mensagem personalizada, além do nome e idade do animal.

    \item Crie uma lista contendo pelo menos \textbf{uma instância de cada tipo de animal}.

    \item Percorra a lista e, para cada animal, chame:
    \begin{itemize}
        \item O método \texttt{emitir\_som()} para exibir o som específico.
        \item O método \texttt{apresentar()} para mostrar as informações detalhadas.
    \end{itemize}

    \item Demonstre o uso de \textbf{polimorfismo}, mostrando que, mesmo com uma lista genérica de
    \texttt{Animal}, cada subclasse executa seu comportamento específico.
\end{enumerate}

\section{Veículos e Oficina}

Crie um sistema para gerenciar veículos em uma oficina mecânica, utilizando herança e polimorfismo.

\begin{enumerate}
    \item Crie uma \textbf{classe abstrata} chamada \texttt{Veiculo}.  
    Ela deve ter os seguintes atributos:
    \begin{itemize}
        \item \texttt{marca} -- marca do veículo,
        \item \texttt{modelo} -- modelo do veículo,
        \item \texttt{ano} -- ano de fabricação,
        \item \texttt{dataEntrada} -- data em que o veículo deu entrada na oficina.
    \end{itemize}

    A classe deve possuir:
    \begin{itemize}
        \item Um \textbf{método abstrato} chamado \texttt{calcularOrcamento()}, que retorna o valor estimado do conserto.
        \item Um método \texttt{exibirDetalhes()} que mostra as informações básicas do veículo.
    \end{itemize}

    \item Crie as seguintes subclasses que herdem de \texttt{Veiculo}:
    \begin{enumerate}
        \item \textbf{\texttt{Carro}}:
        \begin{itemize}
            \item Deve ter um atributo \texttt{numPortas}.
            \item O método \texttt{calcularOrcamento()} deve retornar um valor baseado no ano do carro e na quantidade de portas, considerando que carros mais antigos e com mais portas têm custo de manutenção maior.
        \end{itemize}

        \item \textbf{\texttt{Moto}}:
        \begin{itemize}
            \item Deve ter um atributo \texttt{cilindrada}.
            \item O método \texttt{calcularOrcamento()} deve retornar um valor baseado na cilindrada, sendo que motos de maior cilindrada têm custo de manutenção mais alto.
        \end{itemize}
    \end{enumerate}

    \item Crie uma lista contendo pelo menos \textbf{3 veículos de tipos variados} (por exemplo, 2 carros e 1 moto).

    \item Percorra a lista e, para cada veículo, chame:
    \begin{itemize}
        \item O método \texttt{exibirDetalhes()} para mostrar as informações básicas,
        \item O método \texttt{calcularOrcamento()} para mostrar o valor estimado do conserto.
    \end{itemize}

    \item Mostre que o mesmo método \texttt{calcularOrcamento()} se comporta de forma diferente para carros e motos, demonstrando \textbf{polimorfismo}.
\end{enumerate}

\section{Sistema de Pagamentos}

Desenvolva um sistema para gerenciar diferentes formas de pagamento em uma loja virtual, utilizando herança e polimorfismo.

\begin{enumerate}
    \item Crie uma \textbf{classe abstrata} chamada \texttt{Pagamento}.  
    Ela deve ter os seguintes atributos:
    \begin{itemize}
        \item \texttt{valor} -- valor total da compra,
        \item \texttt{data} -- data do pagamento,
        \item \texttt{nomeCliente} -- nome do cliente que realizou a compra.
    \end{itemize}

    A classe deve possuir:
    \begin{itemize}
        \item Um \textbf{método abstrato} chamado \texttt{processar()}, que realiza a lógica de processamento do pagamento.
        \item Um método \texttt{exibirRecibo()} que imprime as informações básicas da transação (nome do cliente, valor e data).
    \end{itemize}

    \item Crie as seguintes subclasses que herdem de \texttt{Pagamento}:
    \begin{enumerate}
        \item \textbf{\texttt{CartaoCredito}}:
        \begin{itemize}
            \item Deve ter um atributo \texttt{parcelas} -- quantidade de parcelas.
            \item O método \texttt{processar()} deve calcular o valor final com juros de 2\% por parcela e exibir o valor de cada parcela e o total atualizado.
        \end{itemize}

        \item \textbf{\texttt{Boleto}}:
        \begin{itemize}
            \item Deve ter um atributo \texttt{desconto} -- percentual de desconto.
            \item O método \texttt{processar()} deve aplicar o desconto ao valor original e exibir o valor final.
        \end{itemize}

        \item \textbf{\texttt{Pix}}:
        \begin{itemize}
            \item Não precisa de atributos adicionais.
            \item O método \texttt{processar()} deve apenas confirmar o pagamento instantâneo, exibindo a mensagem de sucesso.
        \end{itemize}
    \end{enumerate}

    \item Crie uma lista contendo pelo menos:
    \begin{itemize}
        \item Um pagamento via cartão de crédito,
        \item Um pagamento via boleto,
        \item Um pagamento via Pix.
    \end{itemize}

    \item Percorra a lista e, para cada pagamento:
    \begin{itemize}
        \item Chame o método \texttt{exibirRecibo()} para mostrar os dados básicos da transação,
        \item Em seguida, chame o método \texttt{processar()} para executar a lógica específica daquela forma de pagamento.
    \end{itemize}

    \item Demonstre \textbf{polimorfismo} mostrando que o mesmo método \texttt{processar()} se comporta de maneira diferente dependendo do tipo de pagamento.
\end{enumerate}


\end{document}
