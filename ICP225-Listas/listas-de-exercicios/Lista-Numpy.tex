\title{Lista de exercicios: Numpy}
\author{Gabriel Rodrigues Caldas de Aquino}
\date{\today}
\begin{document}


\maketitle

\textbf{Faça todos os exercícios utilizando Python e Numpy.}

\section{Resolução de Sistema}
\textbf{Resolva}:
\[
    \begin{cases}
        2x_1 + x_2 - x_3 + x_4 + 2x_5 + x_6 - x_7 + x_8 = 5    \\
        x_1 + 3x_2 + x_3 - 2x_4 + x_5 + x_6 + x_7 - x_8 = 12   \\
        -x_1 + 2x_2 + 4x_3 + x_4 - x_5 + x_6 + 2x_7 + x_8 = 3  \\
        3x_1 - x_2 + 2x_3 + 5x_4 + x_5 - x_6 + x_7 + x_8 = 18  \\
        x_1 + x_2 + x_3 + x_4 + x_5 + x_6 + x_7 + x_8 = 10     \\
        2x_1 - 3x_2 + x_3 + 4x_4 - x_5 + 2x_6 - x_7 + 3x_8 = 7 \\
        x_1 + 4x_2 - 2x_3 + x_4 + 2x_5 - 3x_6 + x_7 + x_8 = 9  \\
        4x_1 + x_2 + x_3 - x_4 + x_5 + 2x_6 + x_7 + 5x_8 = 20
    \end{cases}
\]

\begin{itemize}
    \item Encontre o valor das incógnitas.
    \item Faça a operação com a prova real.
\end{itemize}

\section{Vendas de papelaria}

Abra o arquivo \textbf{vendas\_papelaria.csv} que contém a venda de 10 itens em 50 papelarias em uma planilha. Faça:

\begin{itemize}
    \item Descubra a maior e menor quantidade de cada item vendido considerando todas as papelarias.
    \item Descubra para cada papelaria qual foi a quantidade do item que menos vendeu e do item que mais vendeu.
    \item Calcule o total de cada item vendido considerando todas as papelarias.
    \item Calcule quantos items cada papelaria vendeu.
    \item Calcule a média de vendas de cada item.
    \item Faça a transposição da planilha de vendas.
\end{itemize}

\section{Vendas de papelaria e preços}

Considere o arquivo \textbf{preços\_papelarias.csv} que contém o preço de cada item em cada papelaria em uma planilha. Faça:

\begin{itemize}
    \item Calcule o faturamento de cada item em cada papelaria.
    \item Calcule o faturamento de cada papelaria.
    \item Calcule qual foi o faturamento de cada item considerando todas as papelarias.
\end{itemize}

\section{Soma de Forças no Plano}

Um corpo está sujeito à ação de três forças no plano, dadas em newtons (N):

\[
    \vec{F_1} = (8, 3), \quad
    \vec{F_2} = (-5, 7), \quad
    \vec{F_3} = (4, -6)
\]

\textbf{Tarefas:}

\begin{itemize}
    \item Calcule o vetor resultante.
    \item Determine o módulo do vetor resultante.
    \item Aumente a força resultante por 3 e determine o novo vetor.
    \item Calcule o módulo deste novo vetor.
\end{itemize}

\textbf{Dica}: Use \textit{np.linalg.norm()} para calcular o módulo.

\section{Movimento de um Drone}

Um drone de entrega com massa \(m = 5\ \mathrm{kg}\) está em voo. Ele está sujeito a três forças principais:

\[
    \vec{F_1} = (20, 70), \quad \text{Força dos Propulsores}
\]
\[
    \vec{F_2} = (-30, -5), \quad \text{Força do Vento}
\]
\[
    \vec{g} = (0, -9.8), \quad \text{Gravidade}
\]

\textbf{Dica}: Fgrav = m*g


\textbf{Tarefas:}

\begin{enumerate}
    \item Calcule a Força da Gravidade.
    \item Calcule a Força Resultante:
    \item Calcule a aceleração resultante \(\vec{a} = \vec{R} / m\).
\end{enumerate}

\end{document}
