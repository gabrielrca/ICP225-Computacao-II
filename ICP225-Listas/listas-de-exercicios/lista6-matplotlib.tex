\title{Lista de Exercícios: Introdução ao Matplotlib}
\author{Gabriel Rodrigues Caldas de Aquino}
\date{26/06/2025}

\begin{document}

\maketitle

\section{Gráfico de linha simples}

\textbf{Descrição:}  
Crie um gráfico de linha com os pontos (1, 1), (2, 4), (3, 9), (4, 16).  
Adicione um título ao gráfico e rótulos nos eixos x e y.

\section{Gráfico com marcadores personalizados}

\textbf{Descrição:}  
Represente os pontos (2, 1), (3, 4), (4, 9) com marcadores circulares vermelhos e linha pontilhada.  
Utilize o estilo `'ro--'`, defina o tamanho dos marcadores como 8 e adicione uma grade (\texttt{plt.grid(True)}).

\section{Comparando \texttt{plot()} e \texttt{scatter()}}

\textbf{Descrição:}  
Utilize os pontos (1,2), (2,4), (3,1), (4,3) para plotar um gráfico com \texttt{plot()} e outro com \texttt{scatter()}.  
 Defina os limites dos eixos com \texttt{plt.axis((0, 5, 0, 5))}. Explique a diferença visual entre os dois.


\section{Múltiplas curvas no mesmo gráfico}

\textbf{Descrição:}  
Desenhe as curvas \( y = x^2 + 2x + 3 \) e \( y = x^2 \) no mesmo gráfico.  
Utilize cores e estilos diferentes e adicione legenda.

\section{Salvando o gráfico em imagem}

\textbf{Descrição:}  
Escolha um grafico anterior e salve-o como imagem utilizando o comando \texttt{plt.savefig("grafico.png")}.  
A imagem deve ter resolução de 300 DPI. 

\section{Escala do gráfico}

\textbf{Descrição:}  
Plote no mesmo gráfico as funções \( y = x^2 \), \( y = x^4 \) e \( y = x^{10} \), no intervalo de -2 a 2.  
Use diferentes cores e estilos de linha.  
Adicione legenda, grade e título.  
Explique o que ocorre com a escala dos dados.

\section{Efeito da função \texttt{plt.axis()}}

\textbf{Descrição:}  
Reutilize o gráfico da questão anterior.  
Agora, aplique um recorte nos eixos usando \texttt{plt.axis((-2, 2, 0, 10))}.  
Observe e descreva o que acontece com as curvas mais acentuadas.

\section{Criando uma função própria para plotagem}

\textbf{Descrição:}  
Defina uma função para \( f(x) = x^3 + 2x^2 + 5x + 1 \) em python usando \texttt{def}.  
Use \texttt{np.linspace()} para gerar valores no intervalo de -3 a 3. 
Plote o gráfico da função com rótulos nos eixos e grade.

\section{Anotando o ponto de inflexão de uma função}

\textbf{Descrição:}  
Considere a função \( f(x) = x^3 \).  

Plote a função no intervalo de -5 a 5 e anote o ponto (0, 0) com uma seta, utilizando o comando \texttt{plt.annotate()}.  
A anotação deve conter o texto “ponto de inflexão”.  
Adicione título, rótulos nos eixos e grade.



\section{Vendas de discos de vinil ao longo dos anos}

\textbf{Descrição:}  
O arquivo \texttt{vinyl\_sales.txt} contém dados anuais de vendas de discos de vinil, no formato:

\begin{verbatim}
1973,228
1974,204
1975,164
...
2019,0.336120488
\end{verbatim}

\textbf{Tarefa:}  
Utilize o comando do Python para abrir o arquivo e leia linha a linha.  
Para cada linha, separe o ano e o valor das vendas.  
Armazene os anos em uma lista e os valores das vendas em outra.  

Por fim, plote um gráfico de linha para mostrar a evolução das vendas de discos de vinil ao longo dos anos.  
Adicione título, rótulos nos eixos e grade no gráfico.

\textbf{Referência do dataset:}  
\url{https://www.kaggle.com/datasets/imtkaggleteam/40-years-of-music-industry-sales/}

\section{Vendas de discos de vinil ao longo dos anos (Gráfico de barras)}

\textbf{Descrição:}  
Modifique o exercício anterior para ter um gráfico com barras.

\end{document}
