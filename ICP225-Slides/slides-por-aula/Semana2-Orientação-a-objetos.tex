\title{Programação Orientada a Objetos}

\author{Prof. Gabriel Rodrigues Caldas de Aquino}

\institute
{
    gabrielaquino@ic.ufrj.br\\
    
    Instituto de Computação -
    Universidade Federal do Rio de Janeiro % Your institution for the title page
}
\date{Compilado em: \\ \today} % Date, can be changed to a custom date

%----------------------------------------------------------------------------------------
%    PRESENTATION SLIDES
%----------------------------------------------------------------------------------------

%------------------------------------------------
\section{Revisão Inicial}
%------------------------------------------------

\begin{frame}
    % Print the title page as the first slide
    \titlepage
\end{frame}

\begin{frame}{Programação Orientada a Objetos}

\begin{block}{O que é um Objeto?}
\begin{itemize}
    \item Unidade de software que \textbf{encapsula dados e algoritmos}.
    \item Representa conceitos ou entidades do mundo real ou conceitual.
    \item Relaciona-se com outras entidades, assim como na vida cotidiana.
\end{itemize}
\end{block}

\begin{block}{Por que usar?}
\begin{itemize}
    \item Estruturação clara e organizada do código.
    \item Especialmente útil para \textbf{projetos extensos}.
    \item Facilita o desenvolvimento de sistemas:
    \begin{itemize}
        \item Modulares
        \item Reutilizáveis
        \item De fácil manutenção
    \end{itemize}
\end{itemize}
\end{block}

\end{frame}

\begin{frame}[fragile]{Classe em Python}

\begin{block}{O que é uma Classe?}
\begin{itemize}
    \item Unidade inicial e mínima para código orientado a objetos.
    \item Abstrai conceitos, definições e comportamentos.
    \item Descreve atributos (dados) e métodos (ações) dos objetos.
    \item Objetos são \textbf{instanciados} a partir de classes.
\end{itemize}
\end{block}

\begin{block}{Exemplo em Python}
\begin{verbatim}
class Pessoa:
    def __init__(self, nome, idade):
        self.nome = nome
        self.idade = idade
    def apresentar(self):
        print(f"Olá, meu nome é {self.nome} e tenho {self.idade} anos.")
p1 = Pessoa("Maria", 20)
p1.apresentar()
\end{verbatim}
\end{block}

\end{frame}

\begin{frame}[fragile]{Exemplo em Python - Personagem de RPG}

\begin{verbatim}
class Personagem:
    def __init__(self, vida, forca, inteligencia):
        self.vida = vida
        self.forca = forca
        self.inteligencia = inteligencia
    def atacar(self):
        print("O personagem atacou!")
    def defender(self):
        print("O personagem defendeu!")
    def usar_magia(self):
        print("O personagem lançou uma magia!")
# Criando um objeto (instância) da classe
heroi = Personagem(100, 20, 15)
heroi.atacar()
print("Vida:", heroi.vida)
\end{verbatim}

\end{frame}


\begin{frame}[fragile]{Exemplo com Conceitos de POO}

\begin{itemize}
    \item \textbf{Classe}:
Projeto de um personagem de RPG.
\item \textbf{Objeto}:
Cada personagem criado com base nesse projeto.
\end{itemize}

\begin{itemize}
    \item \textbf{Atributos}
    \begin{itemize}
    \item Vida
    \item Força
    \item Inteligência 
    \end{itemize}
    \item \textbf{Métodos}
    \begin{itemize}
    \item Atacar
    \item Defender
    \item Usar magia
\end{itemize}
    
\end{itemize}

\begin{exampleblock}{Trecho de Código}
\begin{verbatim}
heroi = Personagem(100, 20, 15)
heroi.atacar()
\end{verbatim}
\end{exampleblock}

\end{frame}



\begin{frame}{Atributos em POO}

\begin{block}{Definição}
\begin{itemize}
    \item Representam as características do objeto.
    \item Definidos dentro da classe.
    \item Armazenam valores que caracterizam o objeto.
\end{itemize}
\end{block}

\begin{exampleblock}{Exemplo: Classe \texttt{Personagem}}
\begin{itemize}
    \item \textbf{vida} – Pontos de vida do personagem.
    \item \textbf{força} – Poder de ataque físico.
    \item \textbf{inteligência} – Capacidade para magia ou estratégias.
\end{itemize}
\end{exampleblock}

\begin{block}{Observação}
Cada objeto criado a partir da classe \texttt{Personagem} terá seus próprios valores para esses atributos.
\end{block}

\end{frame}

\begin{frame}[fragile]{Métodos em POO}

\begin{block}{Definição}
\begin{itemize}
    \item São as ações que o objeto pode executar.
    \item Definidos como funções dentro da classe.
    \item Podem usar e alterar os atributos do objeto.
\end{itemize}
\end{block}

\begin{exampleblock}{Exemplo: Classe \texttt{Personagem}}
\begin{itemize}
    \item \textbf{atacar()} – Realiza um ataque.
    \item \textbf{defender()} – Executa defesa.
    \item \textbf{usar\_magia()} – Lança um feitiço.
\end{itemize}
\end{exampleblock}

\begin{exampleblock}{Trecho de Código}
\begin{verbatim}
heroi.defender()
heroi.usar_magia()
\end{verbatim}
\end{exampleblock}

\end{frame}

\begin{frame}[fragile]{Método Construtor em Python}

\begin{block}{Definição}
\begin{itemize}
    \item Método especial chamado automaticamente ao criar um objeto.
    \item Usado para inicializar os atributos da nova instância.
    \item Em Python, o construtor é o método \texttt{\_\_init\_\_()}.
\item O construtor:
    \begin{itemize}
    \item Facilita a inicialização de atributos obrigatórios.
    \item Se não for definido, Python usa um construtor default que não faz nada.
\end{itemize}
\end{itemize}
\end{block}

\begin{exampleblock}{Exemplo simples}
\begin{verbatim}
class Personagem:
    def __init__(self, vida, forca):
        self.vida = vida
        self.forca = forca

heroi = Personagem(100, 20)
\end{verbatim}
\end{exampleblock}

\end{frame}

\begin{frame}[fragile]{O parâmetro \texttt{self} em Métodos}

\begin{block}{Definição}
\begin{itemize}
    \item Deve ser o primeiro parâmetro em métodos de instância.
    \item Representa o próprio objeto que chama o método.
    \item Permite acessar atributos e outros métodos do objeto.
    \item É obrigatório explicitar no código, mas não na chamada do método.
\end{itemize}
\end{block}

\begin{exampleblock}{Exemplo simples}
\begin{verbatim}
class Personagem:
    def apresentar(self):
        print("Eu sou um personagem!")

p = Personagem()
p.apresentar()  # 'self' é passado automaticamente
\end{verbatim}
\end{exampleblock}

\end{frame}
\begin{frame}[fragile]{Classe Personagem - Método \texttt{defender}}

\begin{verbatim}
class Personagem:
    def __init__(self, vida, forca, inteligencia):
        self.vida = vida
        self.forca = forca
        self.inteligencia = inteligencia
    def defender(self, dano_recebido):
        dano_final = dano_recebido / 2
        self.vida -= dano_final
        print(f"{self} defendeu e recebeu {dano_final} de dano. Vida restante: {self.vida}")
    def __str__(self):
        return "Personagem"

heroi = Personagem(100, 20, 15)
heroi.defender(30)
print("Vida do heroi:", heroi.vida)
\end{verbatim}

\end{frame}

\begin{frame}[fragile]{Classe Personagem - Método \texttt{ganhar\_healthpack}}

\begin{verbatim}
class Personagem:
    def __init__(self, vida, forca, inteligencia):
        self.vida = vida
        self.forca = forca
        self.inteligencia = inteligencia
    def defender(self, dano_recebido):
        dano_final = dano_recebido / 2
        self.vida -= dano_final
        print(f"{self} defendeu e recebeu {dano_final} de dano. Vida restante: {self.vida}")
    def ganhar_healthpack(self, vida_restaurada):
        self.vida += vida_restaurada
        print(f"{self} recebeu {vida_restaurada} de Healthpack. Vida restante: {self.vida}")
    def __str__(self):
        return "Personagem"
heroi = Personagem(100, 20, 15)
heroi.ganhar_healthpack(50)
\end{verbatim}

\end{frame}

\begin{frame}[fragile]{Deixando mais pessoal - Personagem com nome}

\begin{verbatim}
class Personagem:
    def __init__(self, nome, vida, forca, inteligencia):
        self.nome = nome
        self.vida = vida
        self.forca = forca
        self.inteligencia = inteligencia

    def __str__(self):
        return self.nome

heroi = Personagem("Arthur", 100, 20, 15)
\end{verbatim}

\end{frame}

\begin{frame}[fragile]{Por que o \texttt{print} mostra o nome do personagem?}

\begin{block}{O método \texttt{\_\_str\_\_()}}
\begin{itemize}
    \item O Python usa o método especial \texttt{\_\_str\_\_()} para definir como representar um objeto como texto.
    \item Quando usamos \texttt{print(objeto)}, o Python chama \texttt{objeto.\_\_str\_\_()} automaticamente.
    \item No exemplo, \texttt{\_\_str\_\_()} foi definido para retornar o nome do personagem.
    \item Por isso, o \texttt{print(heroi)} exibe o nome ao invés do endereço de memória.
\end{itemize}
\end{block}

\begin{exampleblock}{Trecho do código}
\begin{verbatim}
def __str__(self):
    return self.nome
\end{verbatim}
\end{exampleblock}

\end{frame}

\begin{frame}[fragile]{Método \texttt{atacar} com interação entre objetos}

\begin{verbatim}
class Personagem:
    def __init__(self, nome, vida, forca, inteligencia):
        self.nome = nome
        self.vida = vida
        self.forca = forca
        self.inteligencia = inteligencia
    def atacar(self, alvo):
        dano = self.forca * 2
        alvo.vida -= dano
        print(f"{self} atacou {alvo} causando {dano} de dano!")
        print(f"Vida restante de {alvo}: {alvo.vida}")
    def __str__(self):
        return self.nome
heroi = Personagem("Arthur", 100, 20, 15)
inimigo = Personagem("Goblin", 80, 15, 10)
heroi.atacar(inimigo)
\end{verbatim}

\end{frame}

\begin{frame}{Discussão em Grupo: Modelando Tipos de Dano}

\begin{block}{Desafio}
Como podemos implementar diferentes tipos de dano em nosso jogo?  

\begin{itemize}
    \item Dano pode ser igual à força do personagem.
    \item Pode ser o dobro (exemplo: fogo contra planta).
    \item Pode ser metade (exemplo: água contra planta).
    \item Ou igual para ataques contra o mesmo tipo.
\end{itemize}


\begin{block}{Pontos para pensar}
\begin{itemize}
    \item Como representar os tipos dos personagens e dos ataques?
    \item Como aplicar multiplicadores diferentes para cada tipo de dano?
    \item Qual estrutura de dados usar para armazenar essas regras?
    \item Como modificar o método \texttt{atacar} para considerar isso?
\end{itemize}
\end{block}

\end{block}

\end{frame}


\begin{frame}[fragile]{Tipos de dano}


\begin{verbatim}
class Personagem:
    def __init__(self, nome, vida, forca, inteligencia, tipo):
        (...)
        self.tipo = tipo  # 'fogo', 'planta', 'agua'
    def atacar(self, alvo):
        if self.tipo == "fogo" and alvo.tipo == "planta":
            dano = self.forca * 2
        elif self.tipo == "agua" and alvo.tipo == "planta":
            dano = self.forca * 0.5
        else:
            dano = self.forca
        alvo.vida -= dano
        print(f"{self} atacou {alvo} causando {dano} de dano!")
        print(f"Vida restante de {alvo}: {alvo.vida}")
heroi = Personagem("Arthur", 100, 20, 15, "fogo")
inimigo = Personagem("Treant", 80, 15, 10, "planta")
\end{verbatim}



\end{frame}
 \begin{frame}{Exercício: Modelando Formas Geométricas}

\textbf{Desafio:} Crie classes em Python para representar formas geométricas básicas.

\begin{itemize}
\item Comece com duas formas: \texttt{Retangulo} e \texttt{Circulo}.
\item Cada classe deve ter:
\begin{itemize}
\item Atributos para as dimensões (ex: base e altura para retângulo, raio para círculo).
\item Método para calcular a área.
\item Método para calcular o perímetro (ou circunferência no caso do círculo).
\end{itemize}
\item Crie objetos de cada classe e imprima a área e perímetro.
\end{itemize}


\end{frame}

\begin{frame}[fragile]{Exemplo: Classe Retângulo}

\begin{block}{Classe \texttt{Retangulo}}
\begin{verbatim}
class Retangulo:
def init(self, base, altura):
self.base = base
self.altura = altura
def area(self):
    return self.base * self.altura
def perimetro(self):
    return 2 * (self.base + self.altura)
\end{verbatim}
\end{block}
\end{frame}
\begin{frame}[fragile]{Exemplo: Classe Círculo}

\begin{block}{Classe \texttt{Circulo}}
\begin{verbatim}
class Circulo:
def init(self, raio):
self.raio = raio
def area(self):
    return 3.1416 * self.raio ** 2

def perimetro(self):
    return 2 * 3.1416 * self.raio
\end{verbatim}
\end{block}

\end{frame}
