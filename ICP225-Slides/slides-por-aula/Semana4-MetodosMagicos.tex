\title{POO - Métodos Mágicos}

\author{Prof. Gabriel Rodrigues Caldas de Aquino}

\institute
{
    gabrielaquino@ic.ufrj.br\\
    
    Instituto de Computação -
    Universidade Federal do Rio de Janeiro % Your institution for the title page
}
\date{Compilado em: \\ \today} % Date, can be changed to a custom date

%----------------------------------------------------------------------------------------
%    PRESENTATION SLIDES
%----------------------------------------------------------------------------------------

%------------------------------------------------
\section{Revisão Inicial}
%------------------------------------------------

\begin{frame}
    % Print the title page as the first slide
    \titlepage
\end{frame}

\begin{frame}{Métodos Mágicos (Dunder)}

\begin{block}{Definição}
Métodos Mágicos são métodos pré-definidos que podem ser sobrescritos para definir como os objetos de uma classe irão interagir com operadores, funções específicas e instruções da linguagem Python.
\end{block}

\begin{block}{Características}
\begin{itemize}
    \item O nome dos métodos mágicos \textbf{inicia e termina com dois underlines} (\_\_).
    \item Também são chamados de \textbf{métodos Dunder} (“Double Underscore”).
    \item Permitem personalizar comportamento de operadores e funções nativas, como:
    \begin{itemize}
        \item \_\_str\_\_ → define como o objeto é convertido em string
        \item \_\_add\_\_ → define o comportamento do operador +
    \end{itemize}
\end{itemize}
\end{block}

\end{frame}

\begin{frame}[fragile]{Método Mágico \_\_str\_\_}

\begin{block}{Relembrando}
O método \texttt{\_\_str\_\_} é um dos métodos mágicos que já vimos e é usado para definir como um objeto será convertido em string.  
Ele é chamado automaticamente quando usamos o \texttt{print} com um objeto.
\end{block}

\begin{exampleblock}{Exemplo}
\begin{verbatim}
class Pessoa:
    def __init__(self, nome, idade):
        self.nome = nome
        self.idade = idade
    def __str__(self):
        return f"{self.nome}, {self.idade} anos"
p = Pessoa("Maria", 20)
print(p)   # chama automaticamente p.__str__()
# Saída: Maria, 20 anos
\end{verbatim}
\end{exampleblock}

\end{frame}


\begin{frame}{Alguns Métodos Mágicos (Dunder)}

\begin{columns}[T] % T para alinhamento no topo

\begin{column}{0.48\textwidth}
\begin{block}{Operadores Aritméticos}
\begin{tabular}{ll}
\texttt{\_\_add\_\_(self, other)} & + \\
\texttt{\_\_sub\_\_(self, other)} & - \\
\texttt{\_\_mul\_\_(self, other)} & * \\
\texttt{\_\_truediv\_\_(self, other)} & / \\
\texttt{\_\_floordiv\_\_(self, other)} & // \\
\texttt{\_\_mod\_\_(self, other)} & \% \\
\texttt{\_\_pow\_\_(self, other)} & **
\end{tabular}
\end{block}
\end{column}

\begin{column}{0.48\textwidth}
\begin{block}{Operadores de Comparação e Condicionais}
\begin{tabular}{ll}
\texttt{\_\_eq\_\_(self, other)} & == \\
\texttt{\_\_ne\_\_(self, other)} & != \\
\texttt{\_\_lt\_\_(self, other)} & $<$ \\
\texttt{\_\_le\_\_(self, other)} & $<=$ \\
\texttt{\_\_gt\_\_(self, other)} & $>$ \\
\texttt{\_\_ge\_\_(self, other)} & $>=$ \\
\texttt{\_\_contains\_\_(self, other)} & in
\end{tabular}
\end{block}
\end{column}

\end{columns}

\end{frame}



\begin{frame}[fragile]{Sobrescrita do operador + (\_\_add\_\_)}

\begin{verbatim}
class ContaBancaria:
    def __init__(self, banco, saldo):
        self.banco = banco
        self.saldo = saldo
    def __add__(self, outra):
        return self.saldo + outra.saldo
    def __str__(self):
        return f"{self.banco}: R$ {self.saldo}"
c1 = ContaBancaria("Santander", 1000)
c2 = ContaBancaria("Itau", 500)
print(c1 + c2)   # 1500
\end{verbatim}


\end{frame}
\begin{frame}[fragile]{Sobrescrita do operador  $>$ (\_\_gt\_\_)}


\begin{verbatim}
class ContaBancaria:
    def __init__(self, banco, saldo):
        self.banco = banco
        self.saldo = saldo
    def __gt__(self, outra):
        return self.saldo > outra.saldo
    def __str__(self):
        return f"{self.banco}: R$ {self.saldo}"
c1 = ContaBancaria("Bradesco", 1000)
c2 = ContaBancaria("Banco do Brasil", 500)
if c1 > c2:
    print(f"A conta {c1.banco} tem mais dinheiro")
else:
    print(f"A conta {c2.banco} tem mais dinheiro")

\end{verbatim}

\end{frame}

\begin{frame}{Alguns Outros Métodos Mágicos (Dunder)}

\begin{block}{Funções Comuns}
\begin{tabular}{ll}
\texttt{\_\_repr\_\_(self)} & \texttt{repr()} \\
\texttt{\_\_str\_\_(self)} & \texttt{str()} \\
\texttt{\_\_len\_\_(self)} & \texttt{len()} \\
\texttt{\_\_abs\_\_(self)} & \texttt{abs()} \\
\end{tabular}
\end{block}

\begin{block}{Operadores em Objetos Indexáveis}
\begin{tabular}{ll}
\texttt{\_\_getitem\_\_(self, key)} & \texttt{x = obj[key]} \\
\texttt{\_\_setitem\_\_(self, key, value)} & \texttt{obj[key] = value} \\
\texttt{\_\_delitem\_\_(self, key)} & \texttt{del obj[key]} \\
\end{tabular}
\end{block}

\begin{block}{Instruções Especiais}
\begin{tabular}{ll}
\texttt{\_\_iter\_\_(self), \_\_next\_\_(self)} & \texttt{for} \\
\texttt{\_\_enter\_\_(self), \_\_exit\_\_(self, exc\_type, exc\_value, traceback)} & \texttt{with} \\
\end{tabular}
\end{block}

\end{frame}

\begin{frame}[fragile]{Exemplo de \_\_getitem\_\_}

\begin{exampleblock}{Classe que armazena notas de uma disciplina}
\begin{verbatim}
class  Notas_disciplina:
    def __init__(self, disciplina, notas):
        self.disciplina = disciplina
        self.notas = notas  # lista de notas

    def __getitem__(self, indice):
        # permite acessar notas como se fosse uma lista
        return self.notas[indice]

Comp2 = Notas_disciplina("Comp2", [7, 8, 9, 10])

print(Comp2[0])  # 7
print(Comp2[2])  # 9
\end{verbatim}
\end{exampleblock}




\end{frame}

