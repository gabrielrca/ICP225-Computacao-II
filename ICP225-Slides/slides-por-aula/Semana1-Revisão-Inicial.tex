\title{Revisão Inicial}

\author{Prof. Gabriel Rodrigues Caldas de Aquino}

\institute
{
    gabrielaquino@ic.ufrj.br\\
    
    Instituto de Computação -
    Universidade Federal do Rio de Janeiro % Your institution for the title page
}
\date{Compilado em: \\ \today} % Date, can be changed to a custom date

%----------------------------------------------------------------------------------------
%    PRESENTATION SLIDES
%----------------------------------------------------------------------------------------

%------------------------------------------------
\section{Revisão Inicial}
%------------------------------------------------

\begin{frame}
    % Print the title page as the first slide
    \titlepage
\end{frame}

\begin{frame}{Material de Referência}

\begin{block}{Site de Computação 1 - Python}
Acesse o conteúdo, exemplos e exercícios do curso de Computação 1 em:

\centering
\href{https://python.ic.ufrj.br/index.html}{\textbf{https://python.ic.ufrj.br/index.html}}

\vspace{1em}

Explore o material complementar, slides, desafios e mais recursos úteis para aprender Python.
\end{block}

\end{frame}



\begin{frame}[fragile]{Paradigmas de Programação}


\begin{block}{Paradigma de Programação}
\begin{itemize}
    \item \textbf{Padrão} para estruturação do programa.
    \item Organização da \textbf{solução algorítmica}.
    \item Exemplo em Python (imperativo estruturado):
    \begin{verbatim}
    def calcular_media(notas):
        soma = 0
        for n in notas:
            soma += n
        return soma / len(notas)
    \end{verbatim}
\end{itemize}
\end{block}
\end{frame}
\begin{frame}[fragile]{Estilos de Programação}
\begin{block}{Estilo de Programação}
\begin{itemize}
    \item \textbf{Organização do código} (legibilidade, manutenção).
    \item Influência no \textbf{raciocínio do programador}.
    \item Exemplo modular em Python:
    \begin{verbatim}
    def quadrado(x):
        return x ** 2

    print(quadrado(5))  # Saída: 25
    \end{verbatim}
\end{itemize}
\end{block}

\centering
\begin{itemize}
    \item Em Comp2: Evolução para POO (classes, herança) e outros paradigmas
 
\end{itemize}
\end{frame}



\begin{frame}[fragile]{Paradigma Imperativo}

\begin{block}{Paradigma Imperativo Estruturado}
\begin{itemize}
    \item Código sequencial
    \item "Faça isso, depois isso, depois aquilo…"
\end{itemize}
\begin{verbatim}
valor = int(input("Digite um número: "))
if valor > 0:
    print("Positivo")
    resultado = valor * 2
else:
    print("Negativo")
    resultado = valor + 10
print("Resultado:", resultado)
\end{verbatim}
\vspace{-0.3cm}

\end{block}
\end{frame}

\begin{frame}[fragile]{Estilo Modular}
\begin{block}{Estilo Modular}
\begin{itemize}
    \item Construção de módulos
    \item Pequenos códigos (módulos) que tem uma função única
    \item Evitar fazer códigos grandes
    \item A combinação dos módulos realiza uma grande tarefa
    \item Os módulos são as funções

\end{itemize}
\begin{verbatim}
# Módulo de processamento
def calcular(x):
    return x * 2 if x > 0 else x + 10
# Módulo de entrada e saída
def ler_imprimir():
    valor = int(input("Digite um número: "))
    print("Resultado:", calcular(valor))
ler_imprimir()
\end{verbatim}
\vspace{-0.3cm}

\end{block}


\end{frame}



\begin{frame}[fragile]{Tipos de Dados Básicos}
\begin{columns}[T]

\column{0.48\textwidth}
\begin{block}{Numéricos}
\begin{verbatim}
# Inteiro (int)
idade = 25

# Ponto flutuante (float)
peso = 68.5
\end{verbatim}
\end{block}

\column{0.48\textwidth}
\begin{block}{Texto (string)}
\begin{verbatim}
# String (str)
nome = "Maria"
msg = 'Olá, mundo!'
\end{verbatim}
\end{block}

\end{columns}

\begin{alertblock}{Características}
\begin{itemize}
    \item \texttt{int}: Valores inteiros (ex: -3, 0, 42)
    \item \texttt{float}: Decimais (ex: 3.14, -0.5)
    \item \texttt{str}: Texto (entre aspas simples ou duplas)
\end{itemize}
\end{alertblock}
\end{frame}

\begin{frame}[fragile]{Operações com Tipos Básicos}
\begin{columns}[T]

\column{0.48\textwidth}
\begin{block}{Numéricas}
\begin{verbatim}
soma = 5 + 3.2         #Result: 8.2
subtracao = 10 - 4     #Result: 6
multiplicacao = 7 * 2  #Result: 14
divisao = 15 / 2       #Result: 7.5
modulo = 10 % 3        #Result: 1
exponenciacao = 2 ** 3 #Result: 8
\end{verbatim}
\end{block}

\column{0.48\textwidth}
\begin{block}{Strings}
\begin{verbatim}
"a" + "b"   # "ab" (concatenação)
"oi" * 3    # "oioioi" (repetição)
len("abc")  # 3 (tamanho)
\end{verbatim}
\end{block}

\end{columns}

\begin{exampleblock}{Importante}
\begin{itemize}
    \item Não misture tipos! \texttt{"10" + 5} → Erro!
    \item Converta com \texttt{int()}, \texttt{str()}, etc.
\end{itemize}
\end{exampleblock}
\end{frame}

\begin{frame}[fragile]{Variáveis e Atribuição}

\begin{block}{O que são variáveis?}
\begin{itemize}
    \item \textbf{Armazenam informações} para uso futuro e são mantidas na \textbf{memória RAM}
    \item Acessadas através de um \textbf{identificador} (nome)
\end{itemize}
\end{block}

\begin{exampleblock}{Comando de Atribuição}
\begin{verbatim}
# Sintaxe: nome_variavel = valor
idade = 25             # Criando variável 'idade'
pi = 3.1415            # Atribuição de valor decimal
mensagem = "Olá!"      # Variável do tipo string
\end{verbatim}
\end{exampleblock}

\begin{alertblock}{Regras Importantes}
\begin{itemize}
    \item Nomes à \textbf{esquerda} do \texttt{=}, valores à \textbf{direita}
    \item Podem ser \textbf{reescritas}: \texttt{contador = 10} → \texttt{contador = 20}
    \item Tipos são \textbf{inferidos} automaticamente
\end{itemize}
\end{alertblock}
\end{frame}

\begin{frame}[fragile]{Funções em Python}


\begin{columns}
    \column{0.60\textwidth}
    \begin{block}{O que são funções?}
        \begin{itemize}
            \item \textbf{Trechos de código} com um objetivo específico
            \item \textbf{Organizam} e \textbf{dividem} problemas complexos
            \item Podem \textbf{retornar valores} ou executar ações
        \end{itemize}
    \end{block}
    
    \column{0.40\textwidth}
    \begin{alertblock}{Por que usar funções?}
        \begin{itemize}
            \item \textbf{Evita repetição} de código
            \item Facilita \textbf{manutenção}
            \item Permite \textbf{reuso} em diferentes partes do programa
        \end{itemize}
    \end{alertblock}
\end{columns}


\begin{exampleblock}{Sintaxe Básica}
\begin{verbatim}
def nome_funcao(argumentos):
    # Corpo da função
    return resultado
# Exemplo real:
def calcular_media(nota1, nota2):
    media = (nota1 + nota2) / 2
    return media
\end{verbatim}
\end{exampleblock}


\end{frame}

\begin{frame}[fragile]{Exemplo Prático de Funções com Reuso}



    \begin{block}{Código em Python}
\begin{verbatim}
def calcular_area(largura, altura):
    """Calcula área de um retângulo"""
    return largura * altura

# Reuso da função
parede = calcular_area(3, 2.5)
janela = calcular_area(1.2, 1.5)
porta = calcular_area(0.8, 2.1)

area_total = parede - janela - porta

print(f"Área para pintar: {area_total}m²")
\end{verbatim}
    \end{block}




\end{frame}

\begin{frame}[fragile]{Escopo de Variáveis em Funções - Cuidado!}
    \begin{exampleblock}{Exemplo Prático}
\begin{verbatim}
def calcular():
    x = 10  # Variável local
    print(x)

calcular()
print(x)  # ERRO! x não existe aqui
\end{verbatim}
    \end{exampleblock}
\begin{columns}[T]
    \column{0.5\textwidth}
    \begin{block}{\textbf{Escopo}: Conceito chave que  determina}
        
        \begin{itemize}
            \item \textbf{Tempo de vida}:\\ Quando a variável existe
            \item \textbf{Visibilidade}:\\ Onde a variável pode ser acessada
        \end{itemize}
    \end{block}


    \column{0.5\textwidth}
    \begin{alertblock}{Regras Importantes}
        \begin{itemize}
            \item Variáveis \textbf{dentro} da função:
            \begin{itemize}
                \item Nascem quando a função é chamada
                \item Morrem quando a função termina
                \item São \textbf{invisíveis} fora da função
            \end{itemize}
            \item \textbf{Cuidado}:\\ Tentar acessar fora causa erro
        \end{itemize}
    \end{alertblock}


\end{columns}


\end{frame}

\begin{frame}[fragile]{Strings em Python - Definição e Índices}

\begin{columns}[T]
    \column{0.5\textwidth}
    \begin{block}{Definição e Criação}
        \begin{itemize}
            \item Sequência imutável de caracteres
            \item Podem usar \texttt{' '} ou \texttt{" "}
            \item Exemplos:
\begin{verbatim}
disciplina = 'Computação 1'
nome = "Maria"
\end{verbatim}
        \end{itemize}
    \end{block}

    \begin{block}{Acesso por Índice}
        \begin{itemize}
            \item Posições começam em 0
            \item Índices negativos acessam do final
\begin{verbatim}
print(nome[0])   # 'M'
print(nome[-1])  # 'a'
\end{verbatim}
        \end{itemize}
    \end{block}

    \column{0.5\textwidth}
    \begin{exampleblock}{Tabela de Índices}
        \centering
        \begin{tabular}{|c|c|c|c|c|c|}
            \hline
            \textbf{Índice +} & 0 & 1 & 2 & 3 & 4 \\
            \hline
            \textbf{Caractere} & M & a & r & i & a \\
            \hline
            \textbf{Índice -} & -5 & -4 & -3 & -2 & -1 \\
            \hline
        \end{tabular}
    \end{exampleblock}

    \begin{block}{Imutabilidade de string}
        \begin{itemize}
            \item Strings são \textbf{imutáveis}
            \item Tentar modificar gera erro:
\begin{verbatim}
nome[0] = 'm'  # TypeError
\end{verbatim}
        \end{itemize}
    \end{block}
\end{columns}

\end{frame}


\begin{frame}[fragile]{Strings em Python - Operações Comuns}


    
    \begin{block}{Operações com Strings}
        \begin{itemize}
            \item \textbf{Concatenação (\texttt{+}):} Junta duas strings.
            \item \textbf{Repetição (\texttt{*}):} Repete a string várias vezes.
            \item \textbf{Tamanho (\texttt{len}):} Retorna o número de caracteres.
            \item \textbf{Pertencimento (\texttt{in}):} Verifica se uma substring está contida.
            \item \textbf{Fatiamento (\texttt{:}):} Retorna partes da string (substrings).
        \end{itemize}
    \end{block}


    \begin{block}{Exemplos}
\begin{verbatim}
len(nome)          # 5
nome + ' Silva'    # 'Maria Silva'
'ri' in nome       # True
'-' * 10           # '----------'
nome[1:4]          # 'ari'
\end{verbatim}
    \end{block}


\end{frame}


\begin{frame}[fragile]{Tudo sobre Strings em Python - Demonstração Prática}



\begin{verbatim}
nome = "Ana Clara"
idade = 22

print(nome[0], nome[-1])  # A a

print(nome[4:9])   # Clara
print(nome[:3])    # Ana

print(len(nome))           # 8
print("Ana" in nome)       # True
print(nome + " Silva")     # Ana Clara Silva

print(nome.split())        # ['Ana', 'Clara']

print(f"{nome} tem {idade} anos")  # Ana Clara tem 22 anos
\end{verbatim}
\end{frame}

\begin{frame}[fragile]{Tuplas em Python}

\begin{block}{Características das Tuplas}
\begin{itemize}
    \item Sequência de dados \textbf{ordenados}, representadas por \textbf{parênteses}.
    \item Podem conter \textbf{diferentes tipos} de dados: inteiros, floats, strings, outras tuplas etc.
    \item São \textbf{imutáveis}: não é possível alterar seus valores após a criação.
    \item São \textbf{indexadas}: acessamos os elementos por índices, como em listas ou strings.
    \item Suportam \textbf{fatiamento} (slicing).
\end{itemize}
\end{block}

\vspace{0.5em}

\begin{block}{Exemplo}
\begin{verbatim}
tuplanum = (1, 2, 3)
mistura = ('Maria', 10, 3.14, (1, 2))

print(tuplanum[0])    # 1
print(mistura[-1])    # (1, 2)
print(tuplanum[1:])   # (2, 3)
\end{verbatim}
\end{block}

\end{frame}


\begin{frame}[fragile]{Listas em Python}

\begin{columns}[T]
    \column{0.55\textwidth}
    \begin{block}{Características das Listas}
        \begin{itemize}
            \item Sequência de valores \textbf{ordenados}, representada por \textbf{colchetes}.
            \item Suporta \textbf{diferentes tipos} de dados: inteiros, floats, strings, outras listas etc.
            \item São \textbf{mutáveis}: é possível alterar, adicionar e remover elementos.
            \item São \textbf{indexadas}, como strings e tuplas.
        \end{itemize}
    \end{block}

    \begin{block}{Exemplo}
\begin{verbatim}
listanum = [1, 2, 3]
listanum[0] = 10
\end{verbatim}
    \end{block}

    \column{0.45\textwidth}
    \begin{exampleblock}{Tabela de Índices}
    \centering
    \begin{tabular}{|c|c|c|c|}
        \hline
        \textbf{Índice +} & 0 & 1 & 2 \\
        \hline
        \textbf{Elemento} & 1 & 2 & 3 \\
        \hline
        \textbf{Índice -} & -3 & -2 & -1 \\
        \hline
    \end{tabular}
    \end{exampleblock}
\end{columns}

\end{frame}

\begin{frame}[fragile]{Listas em Python - Operações Comuns}

\begin{columns}[T]
    \column{0.6\textwidth}
    \begin{block}{Operações com Listas}
        \begin{itemize}
            \item \textbf{Tamanho} (\texttt{len}): número de elementos da lista.
            \item \textbf{Adicionar} (\texttt{append}, \texttt{insert}): adiciona elementos.
            \item \textbf{Remover} (\texttt{remove}, \texttt{pop}): exclui elementos.
            \item \textbf{Concatenação} (\texttt{+}): junta duas listas.
            \item \textbf{Repetição} (\texttt{*}): repete os elementos da lista.
            \item \textbf{Pertencimento} (\texttt{in}): verifica se um valor está na lista.
            \item \textbf{Fatiamento} (\texttt{:}): retorna sublistas.
        \end{itemize}
    \end{block}

    \column{0.4\textwidth}
    \begin{block}{Exemplos}
\begin{verbatim}
lista = [1, 2, 3]
len(lista)     # 3
lista.append(4)# [1, 2, 3, 4]
lista.pop()    # [1, 2, 3]
[0] + lista    # [0, 1, 2, 3]
[1] * 3        # [1, 1, 1]
2 in lista     # True
lista[1:3]     # [2, 3]
\end{verbatim}
    \end{block}
\end{columns}

\end{frame}

\begin{frame}[fragile]{Dicionários em Python}

\begin{block}{Características dos Dicionários}
\begin{itemize}
    \item \textbf{Coleção não ordenada} de dados.
    \item Representada por \texttt{\{ \}} com pares \textbf{chave: valor}.
    \item Acesso aos valores é feito pelas \textbf{chaves}, e não por índices.
    \item Cada \textbf{chave é única} dentro do dicionário.
    \item Os dicionários são \textbf{mutáveis}: é possível alterar, adicionar e remover pares.
\end{itemize}
\end{block}

\vspace{0.5em}

\begin{block}{Exemplo}
\begin{verbatim}
produtos = {
    'farinha': 3.00,
    'feijão': 5.00,
    'leite': 4.25,
    'açúcar': 2.49 }
print(produtos['leite'])  # 4.25
\end{verbatim}
\end{block}

\end{frame}

\begin{frame}[fragile]{Dicionários em Python - Operações Comuns}


    \begin{block}{Principais Operações}
        \begin{itemize}
            \item \textbf{Adicionar ou atualizar elementos}: atribuindo um valor a uma nova chave.
            \item \textbf{Pertencimento} (\texttt{in}): verifica se uma \textbf{chave} está no dicionário.
            \item \textbf{Verificar existência de chave}: com \texttt{in} ou usando \texttt{get()}.
        \end{itemize}
    \end{block}


    \begin{block}{Exemplos}
\begin{verbatim}
produtos['arroz'] = 4.99
'feijão' in produtos       # True
'macarrão' in produtos     # False
preco = produtos.get('leite')
\end{verbatim}
    \end{block}


\end{frame}

\begin{frame}[fragile]{Booleanos em Python}

\begin{block}{O que são Booleanos?}
\begin{itemize}
    \item Tipo \texttt{bool} representa apenas dois valores: \texttt{True} e \texttt{False}.
    \item Usado para expressar condições lógicas.
    \item Muito comum em estruturas de decisão (como \texttt{if}).
\end{itemize}
\end{block}

\vspace{0.5em}

\begin{columns}[T]
    \column{0.5\textwidth}
    \begin{block}{Operadores de Comparação}
    \begin{itemize}
        \item \texttt{==} : Igual a
        \item \texttt{!=} : Diferente de
        \item \texttt{>}  : Maior que
        \item \texttt{>=} : Maior ou igual a
        \item \texttt{<}  : Menor que
        \item \texttt{<=} : Menor ou igual a
    \end{itemize}
    \end{block}

    \column{0.5\textwidth}
    \begin{block}{Exemplos}
\begin{verbatim}
5 > 3        # True
10 == 20     # False
7 != 4       # True
2 <= 2       # True
\end{verbatim}
    \end{block}
\end{columns}

\end{frame}

\begin{frame}[fragile]{Booleanos em Python - Operadores Lógicos}

\begin{block}{Conectando Expressões Booleanas}
\begin{itemize}
    \item \texttt{not} – \textbf{Não lógico}: inverte o valor lógico.
    \item \texttt{and} – \textbf{E lógico}: verdadeiro apenas se as duas expressões forem verdadeiras.
    \item \texttt{or} – \textbf{Ou lógico}: verdadeiro se pelo menos uma das expressões for verdadeira.
\end{itemize}
\end{block}

\vspace{0.5em}

\begin{columns}[T]
    \column{0.5\textwidth}
    \begin{block}{Exemplos}
\begin{verbatim}
not True       # False
not False      # True

True and True  # True
True and False # False
\end{verbatim}
    \end{block}

    \column{0.5\textwidth}
    \begin{block}{Mais Exemplos}
\begin{verbatim}
True or False   # True
False or False  # False

x = 5
x > 0 and x < 10     # True
not (x == 5)         # False
\end{verbatim}
    \end{block}
\end{columns}

\end{frame}

\begin{frame}[fragile]{Estrutura Condicional em Python}

\begin{block}{Decisões com \texttt{if} e \texttt{else}}
\begin{itemize}
    \item Em muitas situações, partes do código devem ser executadas apenas se certas condições forem verdadeiras.
    \item Para isso, usamos expressões booleanas com as estruturas \texttt{if} e \texttt{else}.
    \item O bloco \texttt{if} só será executado se a condição for \texttt{True}.
    \item Caso contrário, o bloco \texttt{else} (se presente) será executado.
\end{itemize}
\end{block}

\vspace{0.5em}

\begin{block}{Sintaxe}
\begin{verbatim}
if <condição>:
    <bloco if>
else:
    <bloco else>
\end{verbatim}
\end{block}

\end{frame}

\begin{frame}[fragile]{Exemplo de Estrutura Condicional}

\begin{block}{Problema}
Queremos verificar se uma pessoa pode votar.  
Se a idade for maior ou igual a 18 anos, ela pode votar.  
Caso contrário, ela não pode votar.
\end{block}

\vspace{1em}

\begin{block}{Código em Python}
\begin{verbatim}
idade = int(input("Digite sua idade: "))

if idade >= 18:
    print("Você pode votar.")
else:
    print("Você não pode votar.")
\end{verbatim}
\end{block}

\end{frame}

\begin{frame}[fragile]{Estrutura de Repetição em Python}

\begin{block}{O que são Loops?}
\begin{itemize}
    \item Permitem executar um trecho de código várias vezes durante a mesma execução.
    \item Úteis para automatizar tarefas repetitivas.
\end{itemize}
\end{block}

\vspace{0.5em}

\begin{block}{Tipos de Estruturas de Repetição}
\begin{itemize}
    \item \textbf{While:} repete enquanto uma condição booleana for verdadeira.
    \item \textbf{For:} itera sobre elementos de uma coleção (strings, listas, tuplas, etc.).
\end{itemize}
\end{block}

\vspace{0.5em}

\begin{block}{Exemplos}
\begin{verbatim}
while condicao:
    # bloco de código
for elemento in iteravel:
    # bloco de código
\end{verbatim}
\end{block}

\end{frame}


\begin{frame}[fragile]{Exemplo de Loop \texttt{while}}

\begin{block}{Contando de 1 a 5 com \texttt{while}}
\begin{verbatim}
contador = 1

while contador <= 5:
    print(contador)
    contador += 1
\end{verbatim}
\end{block}

\end{frame}


\begin{frame}[fragile]{Exemplo de Loop \texttt{for}}

\begin{block}{Iterando sobre uma lista com \texttt{for}}
\begin{verbatim}
nomes = ['Ana', 'Bruno', 'Carlos']

for nome in nomes:
    print(nome)
\end{verbatim}
\end{block}

\end{frame}

\begin{frame}[fragile]{Entrada de Dados em Python}

\begin{block}{Função \texttt{input()}}
\begin{itemize}
    \item Usada para ler dados digitados pelo usuário.
    \item Sempre retorna uma \textbf{string}.
    \item É necessário converter o valor se quisermos um número.
\end{itemize}
\end{block}

\vspace{1em}

\begin{block}{Exemplos}
\begin{verbatim}
nome = input("Digite seu nome: ")
idade = input("Digite sua idade: ")

# Convertendo para inteiro
idade = int(idade)

# Ou diretamente
idade = int(input("Digite sua idade: "))
\end{verbatim}
\end{block}

\end{frame}


\begin{frame}[fragile]{Saída de Dados em Python}

\begin{block}{Função \texttt{print()}}
\begin{itemize}
    \item Utilizada para \textbf{exibir informações} ao usuário.
    \item Pode mostrar textos, resultados de expressões, valores de variáveis, etc.
    \item Aceita múltiplos argumentos separados por vírgula.
\end{itemize}
\end{block}

\vspace{1em}

\begin{block}{Exemplos}
\begin{verbatim}
print("Olá, mundo!")
nome = "Maria"
print("Olá,", nome)

idade = 20
print("Idade:", idade)
\end{verbatim}
\end{block}

\end{frame}
