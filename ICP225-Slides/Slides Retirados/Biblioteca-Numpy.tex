%----------------------------------------------------------------------------------------
%    TITLE PAGE
%----------------------------------------------------------------------------------------

\title{Biblioteca Numpy}

\author{Prof. Gabriel Rodrigues Caldas de Aquino}

\institute
{
    gabrielaquino@ic.ufrj.br\\
    
    Instituto de Computação -
    Universidade Federal do Rio de Janeiro % Your institution for the title page
}
\date{03 de Junho de 2025} % Date, can be changed to a custom date

%----------------------------------------------------------------------------------------
%    PRESENTATION SLIDES
%----------------------------------------------------------------------------------------


\begin{frame}
    % Print the title page as the first slide
    \titlepage
\end{frame}

%------------------------------------------------
\section{Sobre o numpy}
%------------------------------------------------

\begin{frame}{NumPy - Introdução}

\begin{block}{O que é NumPy?}
\begin{itemize}
    \item Pacote fundamental para \textbf{computação científica} em Python
   
    
\end{itemize}
\end{block}

\begin{block}{Objeto principal: \texttt{numpy.ndarray}}
\begin{itemize}
    \item Vetor \textbf{n-dimensional} (arrays multidimensionais)
    \item Características fundamentais:
    \begin{itemize}
        \item \textbf{Tamanho fixo} (definido na criação)
        \item \textbf{Indexado} por tuplas de inteiros positivos
        \item \textbf{Homogêneo} - todos elementos do mesmo tipo
    \end{itemize}
\end{itemize}
\end{block}

\begin{exampleblock}{Documentação Oficial}
\centering

Há diversas métodos e atributos disponíveis no NumPy, os quais podem ser consultados na documentação oficial no endereço:
\url{https://numpy.org/doc/stable/index.html}


\end{exampleblock}


\end{frame}

\begin{frame}[fragile]{NumPy - Arrays}

\begin{block}{Importe o módulo e crie arrays a partir de listas}

    \begin{verbatim}
    import numpy as np
    np.array(lista)
    \end{verbatim}

\end{block}
\begin{columns}[T]
    \begin{column}{0.5\textwidth}
\begin{exampleblock}{Exemplos}
\begin{verbatim}
# Array 1D (vetor)
x = np.array([1, 2, 3])


# Array 2D (matriz)
y = np.array([[1., 0., 0.], 
              [0., 1., 0.]])

\end{verbatim}
\end{exampleblock}
    \end{column}
    
  \begin{column}{0.5\textwidth}  
  \begin{block}{Saída dos Exemplos}
        \begin{verbatim}
#Saída:
>>> x
array([1, 2, 3])

>>> y
array([[1., 0., 0.],
       [0., 1., 0.]])
 \end{verbatim}
        \end{block}
    \end{column}
\end{columns} 
\end{frame}

\begin{frame}{Tipos de Numpy array}

\begin{alertblock}{Verificando o Tipo}
        \begin{verbatim}

>>> type(x)
numpy.ndarray

>>> type(y)
numpy.ndarray
        \end{verbatim}
\end{alertblock}


\begin{block}{Características}
\begin{itemize}
    \item \texttt{x} é um array \textbf{1-dimensional} (vetor)
    \item \texttt{y} é um array \textbf{2-dimensional} (matriz)
    \item Ambos são do tipo \texttt{numpy.ndarray}
\end{itemize}
\end{block}
\end{frame}

\begin{frame}[fragile]{NumPy - O Atributo \texttt{dtype}}

\begin{block}{Definição}
O \texttt{dtype} define o tipo dos elementos armazenados no array NumPy
\end{block}

\begin{exampleblock}{Exemplo Inicial}
\begin{verbatim}
minhalista = [1, 2, 3, 4, 5]
arr = np.array(minhalista)
print(arr)          # array([1, 2, 3, 4, 5])
print(type(arr))    # <class 'numpy.ndarray'>
print(arr.dtype)    # dtype('int64')
\end{verbatim}
\end{exampleblock}

\begin{columns}[T]
    \begin{column}{0.5\textwidth}
        \begin{alertblock}{Com Número Decimal}
\begin{verbatim}
minhalista = [1, 2, 3, 4, 5.5]
arr = np.array(minhalista)
print(arr.dtype)  # dtype('float64')
\end{verbatim}
        \end{alertblock}
    \end{column}
    
    \begin{column}{0.5\textwidth}
        \begin{alertblock}{Com String}
\begin{verbatim}
minhalista = [1, 2, 3, 4, "ola"]
arr = np.array(minhalista)
print(arr.dtype)  # dtype('<U21')
\end{verbatim}
        \end{alertblock}
    \end{column}
\end{columns}


\end{frame}


