


\begin{frame}[fragile]{NumPy - Propriedades de Arrays}

\begin{block}{Criação de Array 2D}
\begin{verbatim}
arr2 = np.array([[1, 2], 
                 [3, 4]])
\end{verbatim}
\end{block}

\begin{columns}[T]
    \begin{column}{0.6\textwidth}
        \begin{exampleblock}{Propriedades Fundamentais}
            \begin{itemize}
                \item \texttt{.ndim} - Número de dimensões
                \item \texttt{.shape} - Tupla com tamanho em cada dimensão
                \item \texttt{.size} - Número total de elementos
                \item \texttt{.dtype} - Tipo dos dados
            \end{itemize}
        \end{exampleblock}
\end{column}
    
    \begin{column}{0.4\textwidth}
        \begin{alertblock}{Aplicado ao Exemplo}
\begin{verbatim}
>>> arr2.ndim
2
>>> arr2.shape
(2, 2)
>>> arr2.size
4
>>> arr2.dtype
dtype('int64')
\end{verbatim}
        \end{alertblock}

    \end{column}
\end{columns}
\end{frame}

\begin{frame}{NumPy - Propriedades Básicas de Arrays}

\centering
\begin{tabular}{|l|p{8cm}|}
\hline
\textbf{Propriedade} & \textbf{Descrição} \\
\hline
\texttt{ndarray.ndim} & Número de eixos (dimensões) do array. \\
\hline
\texttt{ndarray.shape} & Dimensões do array. Uma tupla de inteiros indicando o tamanho em cada dimensão. Para uma matriz com n linhas e m colunas, shape será (n, m). O comprimento da tupla shape é igual ao número de dimensões (\texttt{ndim}). \\
\hline
\texttt{ndarray.size} & Número total de elementos do array. \\
\hline
\texttt{ndarray.dtype} & Objeto que descreve o tipo dos elementos no array. Podem ser usados tipos padrão do Python ou tipos específicos do NumPy como \texttt{numpy.int32}, \texttt{numpy.int16} e \texttt{numpy.float64}. \\
\hline
\end{tabular}






\end{frame}


\begin{frame}[fragile]{NumPy – Arrays com Valores Pré-definidos}

\begin{block}{Por que usar valores pré-definidos?}
\begin{itemize}
    \item Facilita a \textbf{inicialização} de matrizes antes do preenchimento com dados
    \item Evita \textbf{erros} na alocação de memória para grandes arrays
    \item Útil para \textbf{cálculos numéricos} e simulações
   
\end{itemize}
\end{block}

\begin{block}{Funções de Criação}
\begin{tabular}{ll}
\textbf{Função} & \textbf{Descrição} \\
\hline
\texttt{np.zeros(shape)} & Cria array preenchido com \textbf{zeros} \\
\texttt{np.ones(shape)} & Cria array preenchido com \textbf{uns} \\
\texttt{np.empty(shape)} & Cria array com \textbf{valores aleatórios}  \\

\end{tabular}
\end{block}

\end{frame}

\begin{frame}[fragile]{NumPy – Arrays com Valores Pré-definidos}
\begin{columns}[T]
    \begin{column}{0.5\textwidth}
        \begin{exampleblock}{Exemplos 1D}
\begin{verbatim}

# Array de 5 zeros
np.zeros(5)

# Array de 3 uns
np.ones(3)

# Array vazio 4 posições
np.empty(4)
\end{verbatim}
        \end{exampleblock}
    \end{column}
    
    \begin{column}{0.5\textwidth}
        \begin{exampleblock}{Exemplos 2D}
\begin{verbatim}

# Matriz 2x3 de zeros
np.zeros((2,3))

# Matriz 3x3 de uns
np.ones((3,3))

# Matriz 2x2 vazia
np.empty((2,2))
\end{verbatim}
        \end{exampleblock}
    \end{column}
\end{columns}

\begin{alertblock}{Dica Importante}
Especifique sempre o \texttt{dtype} para controle preciso do tipo numérico:\\
\texttt{np.zeros(5, dtype=np.float32)}
\end{alertblock}
\end{frame}

\begin{frame}[fragile]{Exemplo: Criação de Arrays com Zeros}

\begin{verbatim}
# importando o módulo
import numpy as np

# criando um array com cinco elementos sendo zeros
array_zeros = np.zeros(5)
print(array_zeros)
# Saída: [0. 0. 0. 0. 0.] 

# criando uma matriz 3x4 preenchida com zeros
matriz_zeros = np.zeros((3, 4)) 
print(matriz_zeros)
# Saída: [[0. 0. 0. 0.] 
          [0. 0. 0. 0.] 
          [0. 0. 0. 0.]]
\end{verbatim}

\end{frame}
\begin{frame}[fragile]{Exemplo: Criação de Arrays com np.empty}

\begin{verbatim}
# importando o módulo
import numpy as np

# criando uma matriz 3x3 sem inicialização garantida
array_empty = np.empty((3, 3)) 
print(array_empty)
# Saída: [[4.67296746e-307 1.69121096e-306 1.37959131e-306]
          [1.11261162e-306 1.11260619e-306 9.34609790e-307] 
          [8.45559303e-307 9.34600963e-307 1.37959740e-306]]
\end{verbatim}

\begin{alertblock}{Observação}
Os valores podem ser aleatórios, pois \texttt{np.empty} não inicializa os elementos, apenas aloca espaço na memória.
\end{alertblock}

\end{frame}

\begin{frame}[fragile]{Criando Arrays e Matrizes com \texttt{np.arange()}}

\begin{block}{Array 1D (Vetor)}
\begin{verbatim}
import numpy as np

# Vetor de 0 a 8
v = np.arange(9)
print(v)
# [0 1 2 3 4 5 6 7 8]
\end{verbatim}
\end{block}

\begin{columns}[T]
    \begin{column}{0.5\textwidth}
        \begin{block}{Matriz 2x2}
\begin{verbatim}
# Criar e redimensionar
m2x2 = np.arange(1,5).reshape(2,2)
print(m2x2)
# [[1 2]
#  [3 4]]
\end{verbatim}
        \end{block}
    \end{column}
    
    \begin{column}{0.5\textwidth}
        \begin{block}{Matriz 3x3}
\begin{verbatim}
# Criar e redimensionar
m3x3 = np.arange(9).reshape(3,3)
print(m3x3)
# [[0 1 2]
#  [3 4 5]
#  [6 7 8]]
\end{verbatim}
        \end{block}
    \end{column}
\end{columns}


\end{frame}

\begin{frame}[fragile]{NumPy - Entendendo Eixos (Axes)}

\begin{block}{O que são eixos em NumPy?}
Em arrays multidimensionais, os eixos representam as direções ao longo das quais as operações podem ser aplicadas.
\begin{itemize}
    \item \texttt{axis=0}: Linhas (vertical, de cima para baixo)
    \item \texttt{axis=1}: Colunas (horizontal, da esquerda para direita)
\end{itemize}
\end{block}

\begin{exampleblock}{Exemplo Prático}
\begin{verbatim}
import numpy as np
matriz = np.array([[1, 2, 3], [4, 5, 6], [7, 8, 9]])
# Soma ao longo do eixo 0 (colunas)
print(np.sum(matriz, axis=0))
# Saída: [12 15 18]  # (1+4+7, 2+5+8, 3+6+9)
# Soma ao longo do eixo 1 (linhas)
print(np.sum(matriz, axis=1)) 
# Saída: [ 6 15 24]  # (1+2+3, 4+5+6, 7+8+9)
\end{verbatim}
\end{exampleblock}


\end{frame}

\begin{frame}{Operações ao Longo dos Eixos em NumPy}

\centering
\begin{tabular}{|l|p{5cm}|p{5cm}|}
\hline
\textbf{Operação} & \textbf{axis = 0 (colunas)} & \textbf{axis = 1 (linhas)} \\
\hline
\texttt{np.sum()} & Soma por coluna & Soma por linha \\
\hline
\texttt{np.mean()} & Média por coluna & Média por linha \\
\hline
\texttt{np.max()} & Máximo por coluna & Máximo por linha \\
\hline
\texttt{np.min()} & Mínimo por coluna & Mínimo por linha \\
\hline
\end{tabular}


\end{frame}

\begin{frame}[fragile]{Operações Aritméticas em NumPy - Adição}

\begin{block}{Adição Escalar}
\begin{verbatim}
A = np.arange(1, 10)
print(A)
# [1 2 3 4 5 6 7 8 9]
print(A + 5)  # Adição escalar
# [ 6  7  8  9 10 11 12 13 14]
\end{verbatim}
\end{block}


        \begin{exampleblock}{Matriz + Escalar}
\begin{verbatim}
B = np.random.randint(0, 10, (3,3))
print(B)            print(B + 5)  
# [[4 3 1]          # [[ 9  8  6]
#  [6 0 7]          #  [11  5 12]
#  [9 3 3]]         #  [14  8  8]]
\end{verbatim}
        \end{exampleblock}
   


\end{frame}

\begin{frame}[fragile]{Operações entre Matrizes em NumPy - Soma}
\begin{columns}[T]
    \begin{column}{0.5\textwidth}
        \begin{block}{Matriz B}
\begin{verbatim}
B = np.random.randint(0, 10, (3,3))
print(B)
# [[4 3 1]
#  [6 0 7]
#  [9 3 3]]
\end{verbatim}
        \end{block}
\end{column}
    
    \begin{column}{0.5\textwidth}        
        \begin{block}{Matriz C}
\begin{verbatim}
C = np.random.randint(0, 10, (3,3))
print(C)
# [[3 7 1]
#  [6 4 5]
#  [0 1 7]]
\end{verbatim}
        \end{block}
    \end{column}
\end{columns}
    
   
        \begin{exampleblock}{Soma B + C}
\begin{verbatim}
print(B + C)
# [[ 7 10  2]
#  [12  4 12]
#  [ 9  4 10]]
\end{verbatim}
        \end{exampleblock}
        





\end{frame}

\begin{frame}[fragile]{NumPy - Multiplicação escalar e element wise}

        \begin{block}{Multiplicação Escalar}
\begin{verbatim}
A = np.arange(1, 10)
print(A)
# [1 2 3 4 5 6 7 8 9]
print(A * 2)
# [ 2  4  6  8 10 12 14 16 18]
\end{verbatim}
        \end{block}
        
        \begin{block}{Multiplicação Element-wise}
\begin{verbatim}
D = np.arange(11, 20)
print(D)
# [11, 12, 13, 14, 15, 16, 17, 18, 19]
print(A * D)
# [ 11  24  39  56  75  96 119 144 171]
\end{verbatim}
        \end{block}

\end{frame}

\begin{frame}[fragile]{Multiplicação Elemento a Elemento entre Matrizes}


        \begin{block}{Matriz B}
\begin{verbatim}
B = np.array([[4, 3, 1],
              [6, 0, 7],
              [9, 3, 3]])
\end{verbatim}
        \end{block}
        
        \begin{block}{Matriz C}
\begin{verbatim}
C = np.array([[3, 7, 1],
              [6, 4, 5],
              [0, 1, 7]])
\end{verbatim}
        \end{block}

        \begin{exampleblock}{Resultado B * C}
\begin{verbatim}
print(B * C)
# [[12 21  1]
#  [36  0 35]
#  [ 0  3 21]]
\end{verbatim}
        \end{exampleblock}
        

\end{frame}

\begin{frame}[fragile]{Multiplicação Matricial com \texttt{np.dot()}}

\begin{columns}[T]
    \begin{column}{0.4\textwidth}
        \begin{block}{Matriz B (3x3)}
\begin{verbatim}
B = np.array([[4, 3, 1],
              [6, 0, 7],
              [9, 3, 3]])
\end{verbatim}
        \end{block}
        
        \begin{block}{Matriz C (3x3)}
\begin{verbatim}
C = np.array([[3, 7, 1],
              [6, 4, 5],
              [0, 1, 7]])
\end{verbatim}
        \end{block}
    \end{column}
    
    \begin{column}{0.6\textwidth}
        \begin{exampleblock}{Resultado \texttt{np.dot(B, C)}}
\begin{verbatim}
print(np.dot(B, C))
# [[ 30  41  26]
#  [ 18  55  41]
#  [ 45  78  45]]
\end{verbatim}
        \end{exampleblock}
    \end{column}
\end{columns}




\begin{alertblock}{Diferença Fundamental}
\begin{itemize}
    \item \texttt{B * C}: Multiplicação elemento a elemento 
    \item \texttt{np.dot(B,C)}: Multiplicação de matrizes
\end{itemize}
\end{alertblock}
\end{frame}

\begin{frame}[fragile]{Transposição de Matrizes em NumPy}

\begin{block}{Matriz Original}
\begin{verbatim}
B = np.array([[4, 3, 1],
              [6, 0, 7],
              [9, 3, 3]])
\end{verbatim}
\end{block}

\begin{columns}[T]
    \begin{column}{0.5\textwidth}
        \begin{exampleblock}{Método 1: Atributo .T}
\begin{verbatim}
print(B.T)
# [[4 6 9]
#  [3 0 3]
#  [1 7 3]]
\end{verbatim}
        \end{exampleblock}
    \end{column}
    
    \begin{column}{0.5\textwidth}
        \begin{exampleblock}{Método 2: Função transpose()}
\begin{verbatim}
print(B.transpose())
# [[4 6 9]
#  [3 0 3]
#  [1 7 3]]
\end{verbatim}
        \end{exampleblock}
    \end{column}
\end{columns}


\end{frame}